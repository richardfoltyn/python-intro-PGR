\documentclass{scrartcl}

    \usepackage{jupyterlatex}

    \usepackage[breakable]{tcolorbox}
    

    % Basic figure setup, for now with no caption control since it's done
    % automatically by Pandoc (which extracts ![](path) syntax from Markdown).
    \usepackage{graphicx}
    % Maintain compatibility with old templates. Remove in nbconvert 6.0
    \let\Oldincludegraphics\includegraphics
    % Ensure that by default, figures have no caption (until we provide a
    % proper Figure object with a Caption API and a way to capture that
    % in the conversion process - todo).
    \usepackage{caption}
    \DeclareCaptionFormat{nocaption}{}
    \captionsetup{format=nocaption,aboveskip=0pt,belowskip=0pt}

    \usepackage{float}
    \floatplacement{figure}{H} % forces figures to be placed at the correct location
    \usepackage{xcolor} % Allow colors to be defined
    \usepackage{enumerate} % Needed for markdown enumerations to work
    \usepackage{geometry} % Used to adjust the document margins
    \usepackage{amsmath} % Equations
    \usepackage{amssymb} % Equations
    \usepackage{textcomp} % defines textquotesingle
    % Hack from http://tex.stackexchange.com/a/47451/13684:
    \AtBeginDocument{%
        \def\PYZsq{\textquotesingle}% Upright quotes in Pygmentized code
    }
    \usepackage{upquote} % Upright quotes for verbatim code
    \usepackage{eurosym} % defines \euro

    \usepackage{iftex}
    \ifPDFTeX
        \usepackage[T1]{fontenc}
        \IfFileExists{alphabeta.sty}{
              \usepackage{alphabeta}
          }{
              \usepackage[mathletters]{ucs}
              \usepackage[utf8]{inputenc}
          }
    \else
        \usepackage{fontspec}
        \usepackage{unicode-math}
    \fi

    \usepackage{fancyvrb} % verbatim replacement that allows latex
    \usepackage{grffile} % extends the file name processing of package graphics
                         % to support a larger range
    \makeatletter % fix for old versions of grffile with XeLaTeX
    \@ifpackagelater{grffile}{2019/11/01}
    {
      % Do nothing on new versions
    }
    {
      \def\Gread@@xetex#1{%
        \IfFileExists{"\Gin@base".bb}%
        {\Gread@eps{\Gin@base.bb}}%
        {\Gread@@xetex@aux#1}%
      }
    }
    \makeatother
    \usepackage[Export]{adjustbox} % Used to constrain images to a maximum size
    \adjustboxset{max size={0.9\linewidth}{0.9\paperheight}}

    % The hyperref package gives us a pdf with properly built
    % internal navigation ('pdf bookmarks' for the table of contents,
    % internal cross-reference links, web links for URLs, \etc)
    \usepackage{hyperref}
    % The default LaTeX title has an obnoxious amount of whitespace. By default,
    % titling removes some of it. It also provides customization options.
    \usepackage{longtable} % longtable support required by pandoc >1.10
    \usepackage{booktabs}  % table support for pandoc > 1.12.2
    \usepackage{array}     % table support for pandoc >= 2.11.3
    \usepackage{calc}      % table minipage width calculation for pandoc >= 2.11.1
    \usepackage[inline]{enumitem} % IRkernel/repr support (it uses the enumerate* environment)
    \usepackage[normalem]{ulem} % ulem is needed to support strikethroughs (\sout)
                                % normalem makes italics be italics, not underlines
    \usepackage{mathrsfs}
    

    
    % Colors for the hyperref package
    \definecolor{urlcolor}{rgb}{0,.145,.698}
    \definecolor{linkcolor}{rgb}{.71,0.21,0.01}
    \definecolor{citecolor}{rgb}{.12,.54,.11}

    % ANSI colors
    \definecolor{ansi-black}{HTML}{3E424D}
    \definecolor{ansi-black-intense}{HTML}{282C36}
    \definecolor{ansi-red}{HTML}{E75C58}
    \definecolor{ansi-red-intense}{HTML}{B22B31}
    \definecolor{ansi-green}{HTML}{00A250}
    \definecolor{ansi-green-intense}{HTML}{007427}
    \definecolor{ansi-yellow}{HTML}{DDB62B}
    \definecolor{ansi-yellow-intense}{HTML}{B27D12}
    \definecolor{ansi-blue}{HTML}{208FFB}
    \definecolor{ansi-blue-intense}{HTML}{0065CA}
    \definecolor{ansi-magenta}{HTML}{D160C4}
    \definecolor{ansi-magenta-intense}{HTML}{A03196}
    \definecolor{ansi-cyan}{HTML}{60C6C8}
    \definecolor{ansi-cyan-intense}{HTML}{258F8F}
    \definecolor{ansi-white}{HTML}{C5C1B4}
    \definecolor{ansi-white-intense}{HTML}{A1A6B2}
    \definecolor{ansi-default-inverse-fg}{HTML}{FFFFFF}
    \definecolor{ansi-default-inverse-bg}{HTML}{000000}

    % common color for the border for error outputs.
    \definecolor{outerrorbackground}{HTML}{FFDFDF}

    % commands and environments needed by pandoc snippets
    % extracted from the output of `pandoc -s`
    \providecommand{\tightlist}{%
      \setlength{\itemsep}{0pt}\setlength{\parskip}{0pt}}
    \DefineVerbatimEnvironment{Highlighting}{Verbatim}{commandchars=\\\{\}}
    % Add ',fontsize=\small' for more characters per line
    \newenvironment{Shaded}{}{}
    \newcommand{\KeywordTok}[1]{\textcolor[rgb]{0.00,0.44,0.13}{\textbf{{#1}}}}
    \newcommand{\DataTypeTok}[1]{\textcolor[rgb]{0.56,0.13,0.00}{{#1}}}
    \newcommand{\DecValTok}[1]{\textcolor[rgb]{0.25,0.63,0.44}{{#1}}}
    \newcommand{\BaseNTok}[1]{\textcolor[rgb]{0.25,0.63,0.44}{{#1}}}
    \newcommand{\FloatTok}[1]{\textcolor[rgb]{0.25,0.63,0.44}{{#1}}}
    \newcommand{\CharTok}[1]{\textcolor[rgb]{0.25,0.44,0.63}{{#1}}}
    \newcommand{\StringTok}[1]{\textcolor[rgb]{0.25,0.44,0.63}{{#1}}}
    \newcommand{\CommentTok}[1]{\textcolor[rgb]{0.38,0.63,0.69}{\textit{{#1}}}}
    \newcommand{\OtherTok}[1]{\textcolor[rgb]{0.00,0.44,0.13}{{#1}}}
    \newcommand{\AlertTok}[1]{\textcolor[rgb]{1.00,0.00,0.00}{\textbf{{#1}}}}
    \newcommand{\FunctionTok}[1]{\textcolor[rgb]{0.02,0.16,0.49}{{#1}}}
    \newcommand{\RegionMarkerTok}[1]{{#1}}
    \newcommand{\ErrorTok}[1]{\textcolor[rgb]{1.00,0.00,0.00}{\textbf{{#1}}}}
    \newcommand{\NormalTok}[1]{{#1}}

    % Additional commands for more recent versions of Pandoc
    \newcommand{\ConstantTok}[1]{\textcolor[rgb]{0.53,0.00,0.00}{{#1}}}
    \newcommand{\SpecialCharTok}[1]{\textcolor[rgb]{0.25,0.44,0.63}{{#1}}}
    \newcommand{\VerbatimStringTok}[1]{\textcolor[rgb]{0.25,0.44,0.63}{{#1}}}
    \newcommand{\SpecialStringTok}[1]{\textcolor[rgb]{0.73,0.40,0.53}{{#1}}}
    \newcommand{\ImportTok}[1]{{#1}}
    \newcommand{\DocumentationTok}[1]{\textcolor[rgb]{0.73,0.13,0.13}{\textit{{#1}}}}
    \newcommand{\AnnotationTok}[1]{\textcolor[rgb]{0.38,0.63,0.69}{\textbf{\textit{{#1}}}}}
    \newcommand{\CommentVarTok}[1]{\textcolor[rgb]{0.38,0.63,0.69}{\textbf{\textit{{#1}}}}}
    \newcommand{\VariableTok}[1]{\textcolor[rgb]{0.10,0.09,0.49}{{#1}}}
    \newcommand{\ControlFlowTok}[1]{\textcolor[rgb]{0.00,0.44,0.13}{\textbf{{#1}}}}
    \newcommand{\OperatorTok}[1]{\textcolor[rgb]{0.40,0.40,0.40}{{#1}}}
    \newcommand{\BuiltInTok}[1]{{#1}}
    \newcommand{\ExtensionTok}[1]{{#1}}
    \newcommand{\PreprocessorTok}[1]{\textcolor[rgb]{0.74,0.48,0.00}{{#1}}}
    \newcommand{\AttributeTok}[1]{\textcolor[rgb]{0.49,0.56,0.16}{{#1}}}
    \newcommand{\InformationTok}[1]{\textcolor[rgb]{0.38,0.63,0.69}{\textbf{\textit{{#1}}}}}
    \newcommand{\WarningTok}[1]{\textcolor[rgb]{0.38,0.63,0.69}{\textbf{\textit{{#1}}}}}


    % Define a nice break command that doesn't care if a line doesn't already
    % exist.
    \def\br{\hspace*{\fill} \\* }
    % Math Jax compatibility definitions
    \def\gt{>}
    \def\lt{<}
    \let\Oldtex\TeX
    \let\Oldlatex\LaTeX
    \renewcommand{\TeX}{\textrm{\Oldtex}}
    \renewcommand{\LaTeX}{\textrm{\Oldlatex}}
    % Document parameters
    % Document title
    \newcommand*{\unitcounter}{7}
    \addtocounter{section}{\unitcounter}
    \addtocounter{section}{-1}
    \newcommand*{\mytitle}{Unit \unitcounter: Data input and output}

    % Included at the bottom of the preamble

\title{\mytitle}
\author{Richard Foltyn}
\institute{University of Glasgow}
\hypersetup{pdfauthor={Richard Foltyn}, pdftitle={\mytitle}}

% needed to access graphs in lectures/solutions folder
\graphicspath{{../lectures/}}


    
    
    
    
    
% Pygments definitions
\makeatletter
\def\PY@reset{\let\PY@it=\relax \let\PY@bf=\relax%
    \let\PY@ul=\relax \let\PY@tc=\relax%
    \let\PY@bc=\relax \let\PY@ff=\relax}
\def\PY@tok#1{\csname PY@tok@#1\endcsname}
\def\PY@toks#1+{\ifx\relax#1\empty\else%
    \PY@tok{#1}\expandafter\PY@toks\fi}
\def\PY@do#1{\PY@bc{\PY@tc{\PY@ul{%
    \PY@it{\PY@bf{\PY@ff{#1}}}}}}}
\def\PY#1#2{\PY@reset\PY@toks#1+\relax+\PY@do{#2}}

\@namedef{PY@tok@w}{\def\PY@tc##1{\textcolor[rgb]{0.73,0.73,0.73}{##1}}}
\@namedef{PY@tok@c}{\let\PY@it=\textit\def\PY@tc##1{\textcolor[rgb]{0.24,0.48,0.48}{##1}}}
\@namedef{PY@tok@cp}{\def\PY@tc##1{\textcolor[rgb]{0.61,0.40,0.00}{##1}}}
\@namedef{PY@tok@k}{\let\PY@bf=\textbf\def\PY@tc##1{\textcolor[rgb]{0.00,0.50,0.00}{##1}}}
\@namedef{PY@tok@kp}{\def\PY@tc##1{\textcolor[rgb]{0.00,0.50,0.00}{##1}}}
\@namedef{PY@tok@kt}{\def\PY@tc##1{\textcolor[rgb]{0.69,0.00,0.25}{##1}}}
\@namedef{PY@tok@o}{\def\PY@tc##1{\textcolor[rgb]{0.40,0.40,0.40}{##1}}}
\@namedef{PY@tok@ow}{\let\PY@bf=\textbf\def\PY@tc##1{\textcolor[rgb]{0.67,0.13,1.00}{##1}}}
\@namedef{PY@tok@nb}{\def\PY@tc##1{\textcolor[rgb]{0.00,0.50,0.00}{##1}}}
\@namedef{PY@tok@nf}{\def\PY@tc##1{\textcolor[rgb]{0.00,0.00,1.00}{##1}}}
\@namedef{PY@tok@nc}{\let\PY@bf=\textbf\def\PY@tc##1{\textcolor[rgb]{0.00,0.00,1.00}{##1}}}
\@namedef{PY@tok@nn}{\let\PY@bf=\textbf\def\PY@tc##1{\textcolor[rgb]{0.00,0.00,1.00}{##1}}}
\@namedef{PY@tok@ne}{\let\PY@bf=\textbf\def\PY@tc##1{\textcolor[rgb]{0.80,0.25,0.22}{##1}}}
\@namedef{PY@tok@nv}{\def\PY@tc##1{\textcolor[rgb]{0.10,0.09,0.49}{##1}}}
\@namedef{PY@tok@no}{\def\PY@tc##1{\textcolor[rgb]{0.53,0.00,0.00}{##1}}}
\@namedef{PY@tok@nl}{\def\PY@tc##1{\textcolor[rgb]{0.46,0.46,0.00}{##1}}}
\@namedef{PY@tok@ni}{\let\PY@bf=\textbf\def\PY@tc##1{\textcolor[rgb]{0.44,0.44,0.44}{##1}}}
\@namedef{PY@tok@na}{\def\PY@tc##1{\textcolor[rgb]{0.41,0.47,0.13}{##1}}}
\@namedef{PY@tok@nt}{\let\PY@bf=\textbf\def\PY@tc##1{\textcolor[rgb]{0.00,0.50,0.00}{##1}}}
\@namedef{PY@tok@nd}{\def\PY@tc##1{\textcolor[rgb]{0.67,0.13,1.00}{##1}}}
\@namedef{PY@tok@s}{\def\PY@tc##1{\textcolor[rgb]{0.73,0.13,0.13}{##1}}}
\@namedef{PY@tok@sd}{\let\PY@it=\textit\def\PY@tc##1{\textcolor[rgb]{0.73,0.13,0.13}{##1}}}
\@namedef{PY@tok@si}{\let\PY@bf=\textbf\def\PY@tc##1{\textcolor[rgb]{0.64,0.35,0.47}{##1}}}
\@namedef{PY@tok@se}{\let\PY@bf=\textbf\def\PY@tc##1{\textcolor[rgb]{0.67,0.36,0.12}{##1}}}
\@namedef{PY@tok@sr}{\def\PY@tc##1{\textcolor[rgb]{0.64,0.35,0.47}{##1}}}
\@namedef{PY@tok@ss}{\def\PY@tc##1{\textcolor[rgb]{0.10,0.09,0.49}{##1}}}
\@namedef{PY@tok@sx}{\def\PY@tc##1{\textcolor[rgb]{0.00,0.50,0.00}{##1}}}
\@namedef{PY@tok@m}{\def\PY@tc##1{\textcolor[rgb]{0.40,0.40,0.40}{##1}}}
\@namedef{PY@tok@gh}{\let\PY@bf=\textbf\def\PY@tc##1{\textcolor[rgb]{0.00,0.00,0.50}{##1}}}
\@namedef{PY@tok@gu}{\let\PY@bf=\textbf\def\PY@tc##1{\textcolor[rgb]{0.50,0.00,0.50}{##1}}}
\@namedef{PY@tok@gd}{\def\PY@tc##1{\textcolor[rgb]{0.63,0.00,0.00}{##1}}}
\@namedef{PY@tok@gi}{\def\PY@tc##1{\textcolor[rgb]{0.00,0.52,0.00}{##1}}}
\@namedef{PY@tok@gr}{\def\PY@tc##1{\textcolor[rgb]{0.89,0.00,0.00}{##1}}}
\@namedef{PY@tok@ge}{\let\PY@it=\textit}
\@namedef{PY@tok@gs}{\let\PY@bf=\textbf}
\@namedef{PY@tok@gp}{\let\PY@bf=\textbf\def\PY@tc##1{\textcolor[rgb]{0.00,0.00,0.50}{##1}}}
\@namedef{PY@tok@go}{\def\PY@tc##1{\textcolor[rgb]{0.44,0.44,0.44}{##1}}}
\@namedef{PY@tok@gt}{\def\PY@tc##1{\textcolor[rgb]{0.00,0.27,0.87}{##1}}}
\@namedef{PY@tok@err}{\def\PY@bc##1{{\setlength{\fboxsep}{\string -\fboxrule}\fcolorbox[rgb]{1.00,0.00,0.00}{1,1,1}{\strut ##1}}}}
\@namedef{PY@tok@kc}{\let\PY@bf=\textbf\def\PY@tc##1{\textcolor[rgb]{0.00,0.50,0.00}{##1}}}
\@namedef{PY@tok@kd}{\let\PY@bf=\textbf\def\PY@tc##1{\textcolor[rgb]{0.00,0.50,0.00}{##1}}}
\@namedef{PY@tok@kn}{\let\PY@bf=\textbf\def\PY@tc##1{\textcolor[rgb]{0.00,0.50,0.00}{##1}}}
\@namedef{PY@tok@kr}{\let\PY@bf=\textbf\def\PY@tc##1{\textcolor[rgb]{0.00,0.50,0.00}{##1}}}
\@namedef{PY@tok@bp}{\def\PY@tc##1{\textcolor[rgb]{0.00,0.50,0.00}{##1}}}
\@namedef{PY@tok@fm}{\def\PY@tc##1{\textcolor[rgb]{0.00,0.00,1.00}{##1}}}
\@namedef{PY@tok@vc}{\def\PY@tc##1{\textcolor[rgb]{0.10,0.09,0.49}{##1}}}
\@namedef{PY@tok@vg}{\def\PY@tc##1{\textcolor[rgb]{0.10,0.09,0.49}{##1}}}
\@namedef{PY@tok@vi}{\def\PY@tc##1{\textcolor[rgb]{0.10,0.09,0.49}{##1}}}
\@namedef{PY@tok@vm}{\def\PY@tc##1{\textcolor[rgb]{0.10,0.09,0.49}{##1}}}
\@namedef{PY@tok@sa}{\def\PY@tc##1{\textcolor[rgb]{0.73,0.13,0.13}{##1}}}
\@namedef{PY@tok@sb}{\def\PY@tc##1{\textcolor[rgb]{0.73,0.13,0.13}{##1}}}
\@namedef{PY@tok@sc}{\def\PY@tc##1{\textcolor[rgb]{0.73,0.13,0.13}{##1}}}
\@namedef{PY@tok@dl}{\def\PY@tc##1{\textcolor[rgb]{0.73,0.13,0.13}{##1}}}
\@namedef{PY@tok@s2}{\def\PY@tc##1{\textcolor[rgb]{0.73,0.13,0.13}{##1}}}
\@namedef{PY@tok@sh}{\def\PY@tc##1{\textcolor[rgb]{0.73,0.13,0.13}{##1}}}
\@namedef{PY@tok@s1}{\def\PY@tc##1{\textcolor[rgb]{0.73,0.13,0.13}{##1}}}
\@namedef{PY@tok@mb}{\def\PY@tc##1{\textcolor[rgb]{0.40,0.40,0.40}{##1}}}
\@namedef{PY@tok@mf}{\def\PY@tc##1{\textcolor[rgb]{0.40,0.40,0.40}{##1}}}
\@namedef{PY@tok@mh}{\def\PY@tc##1{\textcolor[rgb]{0.40,0.40,0.40}{##1}}}
\@namedef{PY@tok@mi}{\def\PY@tc##1{\textcolor[rgb]{0.40,0.40,0.40}{##1}}}
\@namedef{PY@tok@il}{\def\PY@tc##1{\textcolor[rgb]{0.40,0.40,0.40}{##1}}}
\@namedef{PY@tok@mo}{\def\PY@tc##1{\textcolor[rgb]{0.40,0.40,0.40}{##1}}}
\@namedef{PY@tok@ch}{\let\PY@it=\textit\def\PY@tc##1{\textcolor[rgb]{0.24,0.48,0.48}{##1}}}
\@namedef{PY@tok@cm}{\let\PY@it=\textit\def\PY@tc##1{\textcolor[rgb]{0.24,0.48,0.48}{##1}}}
\@namedef{PY@tok@cpf}{\let\PY@it=\textit\def\PY@tc##1{\textcolor[rgb]{0.24,0.48,0.48}{##1}}}
\@namedef{PY@tok@c1}{\let\PY@it=\textit\def\PY@tc##1{\textcolor[rgb]{0.24,0.48,0.48}{##1}}}
\@namedef{PY@tok@cs}{\let\PY@it=\textit\def\PY@tc##1{\textcolor[rgb]{0.24,0.48,0.48}{##1}}}

\def\PYZbs{\char`\\}
\def\PYZus{\char`\_}
\def\PYZob{\char`\{}
\def\PYZcb{\char`\}}
\def\PYZca{\char`\^}
\def\PYZam{\char`\&}
\def\PYZlt{\char`\<}
\def\PYZgt{\char`\>}
\def\PYZsh{\char`\#}
\def\PYZpc{\char`\%}
\def\PYZdl{\char`\$}
\def\PYZhy{\char`\-}
\def\PYZsq{\char`\'}
\def\PYZdq{\char`\"}
\def\PYZti{\char`\~}
% for compatibility with earlier versions
\def\PYZat{@}
\def\PYZlb{[}
\def\PYZrb{]}
\makeatother


    % For linebreaks inside Verbatim environment from package fancyvrb.
    \makeatletter
        \newbox\Wrappedcontinuationbox
        \newbox\Wrappedvisiblespacebox
        \newcommand*\Wrappedvisiblespace {\textcolor{red}{\textvisiblespace}}
        \newcommand*\Wrappedcontinuationsymbol {\textcolor{red}{\llap{\tiny$\m@th\hookrightarrow$}}}
        \newcommand*\Wrappedcontinuationindent {3ex }
        \newcommand*\Wrappedafterbreak {\kern\Wrappedcontinuationindent\copy\Wrappedcontinuationbox}
        % Take advantage of the already applied Pygments mark-up to insert
        % potential linebreaks for TeX processing.
        %        {, <, #, %, $, ' and ": go to next line.
        %        _, }, ^, &, >, - and ~: stay at end of broken line.
        % Use of \textquotesingle for straight quote.
        \newcommand*\Wrappedbreaksatspecials {%
            \def\PYGZus{\discretionary{\char`\_}{\Wrappedafterbreak}{\char`\_}}%
            \def\PYGZob{\discretionary{}{\Wrappedafterbreak\char`\{}{\char`\{}}%
            \def\PYGZcb{\discretionary{\char`\}}{\Wrappedafterbreak}{\char`\}}}%
            \def\PYGZca{\discretionary{\char`\^}{\Wrappedafterbreak}{\char`\^}}%
            \def\PYGZam{\discretionary{\char`\&}{\Wrappedafterbreak}{\char`\&}}%
            \def\PYGZlt{\discretionary{}{\Wrappedafterbreak\char`\<}{\char`\<}}%
            \def\PYGZgt{\discretionary{\char`\>}{\Wrappedafterbreak}{\char`\>}}%
            \def\PYGZsh{\discretionary{}{\Wrappedafterbreak\char`\#}{\char`\#}}%
            \def\PYGZpc{\discretionary{}{\Wrappedafterbreak\char`\%}{\char`\%}}%
            \def\PYGZdl{\discretionary{}{\Wrappedafterbreak\char`\$}{\char`\$}}%
            \def\PYGZhy{\discretionary{\char`\-}{\Wrappedafterbreak}{\char`\-}}%
            \def\PYGZsq{\discretionary{}{\Wrappedafterbreak\textquotesingle}{\textquotesingle}}%
            \def\PYGZdq{\discretionary{}{\Wrappedafterbreak\char`\"}{\char`\"}}%
            \def\PYGZti{\discretionary{\char`\~}{\Wrappedafterbreak}{\char`\~}}%
        }
        % Some characters . , ; ? ! / are not pygmentized.
        % This macro makes them "active" and they will insert potential linebreaks
        \newcommand*\Wrappedbreaksatpunct {%
            \lccode`\~`\.\lowercase{\def~}{\discretionary{\hbox{\char`\.}}{\Wrappedafterbreak}{\hbox{\char`\.}}}%
            \lccode`\~`\,\lowercase{\def~}{\discretionary{\hbox{\char`\,}}{\Wrappedafterbreak}{\hbox{\char`\,}}}%
            \lccode`\~`\;\lowercase{\def~}{\discretionary{\hbox{\char`\;}}{\Wrappedafterbreak}{\hbox{\char`\;}}}%
            \lccode`\~`\:\lowercase{\def~}{\discretionary{\hbox{\char`\:}}{\Wrappedafterbreak}{\hbox{\char`\:}}}%
            \lccode`\~`\?\lowercase{\def~}{\discretionary{\hbox{\char`\?}}{\Wrappedafterbreak}{\hbox{\char`\?}}}%
            \lccode`\~`\!\lowercase{\def~}{\discretionary{\hbox{\char`\!}}{\Wrappedafterbreak}{\hbox{\char`\!}}}%
            \lccode`\~`\/\lowercase{\def~}{\discretionary{\hbox{\char`\/}}{\Wrappedafterbreak}{\hbox{\char`\/}}}%
            \catcode`\.\active
            \catcode`\,\active
            \catcode`\;\active
            \catcode`\:\active
            \catcode`\?\active
            \catcode`\!\active
            \catcode`\/\active
            \lccode`\~`\~
        }
    \makeatother

    \let\OriginalVerbatim=\Verbatim
    \makeatletter
    \renewcommand{\Verbatim}[1][1]{%
        %\parskip\z@skip
        \sbox\Wrappedcontinuationbox {\Wrappedcontinuationsymbol}%
        \sbox\Wrappedvisiblespacebox {\FV@SetupFont\Wrappedvisiblespace}%
        \def\FancyVerbFormatLine ##1{\hsize\linewidth
            \vtop{\raggedright\hyphenpenalty\z@\exhyphenpenalty\z@
                \doublehyphendemerits\z@\finalhyphendemerits\z@
                \strut ##1\strut}%
        }%
        % If the linebreak is at a space, the latter will be displayed as visible
        % space at end of first line, and a continuation symbol starts next line.
        % Stretch/shrink are however usually zero for typewriter font.
        \def\FV@Space {%
            \nobreak\hskip\z@ plus\fontdimen3\font minus\fontdimen4\font
            \discretionary{\copy\Wrappedvisiblespacebox}{\Wrappedafterbreak}
            {\kern\fontdimen2\font}%
        }%

        % Allow breaks at special characters using \PYG... macros.
        \Wrappedbreaksatspecials
        % Breaks at punctuation characters . , ; ? ! and / need catcode=\active
        \OriginalVerbatim[#1,fontsize=\small,codes*=\Wrappedbreaksatpunct]%
    }
    \makeatother

    % Exact colors from NB
    \definecolor{incolor}{HTML}{303F9F}
    \definecolor{outcolor}{HTML}{D84315}
    \definecolor{cellborder}{HTML}{CFCFCF}
    \definecolor{cellbackground}{HTML}{FCFCFC}

    % prompt
    \makeatletter
    \newcommand{\boxspacing}{\kern\kvtcb@left@rule\kern\kvtcb@boxsep}
    \makeatother
    \newcommand{\prompt}[4]{
        {\ttfamily\llap{{\color{#2}[#3]:\hspace{3pt}#4}}\vspace{-\baselineskip}}
    }
    

    
    % Prevent overflowing lines due to hard-to-break entities
    \sloppy
    % Setup hyperref package
    \hypersetup{
      breaklinks=true,  % so long urls are correctly broken across lines
      colorlinks=true,
      urlcolor=urlcolor,
      linkcolor=linkcolor,
      citecolor=citecolor,
      }
    % Slightly bigger margins than the latex defaults
    
    \geometry{verbose,tmargin=1in,bmargin=1in,lmargin=1in,rmargin=1in}
    
    

\begin{document}
    
    \maketitle
    \tableofcontents
    
    

    
    \hypertarget{data-input-and-output}{%
\section{Data input and output}\label{data-input-and-output}}

In this unit we discuss input and output, or I/O for short. We focus
exclusively on I/O routines used to load and store data from files that
are relevant for numerical computation and data analysis.

    \hypertarget{inputoutput-with-numpy}{%
\subsection{Input/output with NumPy}\label{inputoutput-with-numpy}}

\hypertarget{loading-text-data}{%
\subsubsection{Loading text data}\label{loading-text-data}}

We have already encountered the most basic, and probably most frequently
used NumPy I/O routine, \texttt{np.loadtxt()}. We often use files that
store data as text files containing character-separated values (CSV)
since virtually any application supports this data format. The most
important I/O functions to process text data are:

\begin{itemize}
\tightlist
\item
  \href{https://numpy.org/doc/stable/reference/generated/numpy.loadtxt.html}{\texttt{np.loadtxt()}}:
  load data from a text file.
\item
  \href{https://numpy.org/doc/stable/reference/generated/numpy.genfromtxt.html}{\texttt{np.genfromtxt()}}:
  load data from a text file and handle missing data.
\item
  \href{https://numpy.org/doc/stable/reference/generated/numpy.savetxt.html}{\texttt{np.savetxt()}}:
  save a NumPy array to a text file.
\end{itemize}

There are a few other I/O functions in NumPy, for example to write
arrays as raw binary data. We won't cover them here, but you can find
them in the
\href{https://numpy.org/doc/stable/reference/routines.io.html}{official
documentation}.

\vspace{1em}\emph{Example: Load character-separated text data}

Imagine we have the following tabular data from
\href{https://fred.stlouisfed.org/}{FRED}, where the first two rows look
as follows:

\begin{longtable}[]{@{}llll@{}}
\toprule
Year & GDP & CPI & UNRATE\tabularnewline
\midrule
\endhead
1948 & 2118.5 & 24.0 & 3.8\tabularnewline
1949 & 2106.6 & 23.8 & 6.0\tabularnewline
\bottomrule
\end{longtable}

These data are stored as character-separated values (CSV). To load this
CSV file as a NumPy array, we use
\href{(https://numpy.org/doc/stable/reference/generated/numpy.loadtxt.html)}{\texttt{loadtxt()}}.
As in the previous unit, it is advantageous to globally set the path to
the \texttt{data/} directory that can point either to the local
directory or to the \texttt{data/} directory on GitHub.

    \begin{tcolorbox}[breakable, size=fbox, boxrule=1pt, pad at break*=1mm,colback=cellbackground, colframe=cellborder]
\prompt{In}{incolor}{1}{\boxspacing}
\begin{Verbatim}[commandchars=\\\{\}]
\PY{c+c1}{\PYZsh{} Uncomment this to use files in the local data/ directory}
\PY{n}{DATA\PYZus{}PATH} \PY{o}{=} \PY{l+s+s1}{\PYZsq{}}\PY{l+s+s1}{../data}\PY{l+s+s1}{\PYZsq{}}

\PY{c+c1}{\PYZsh{} Load data directly from GitHub}
\PY{c+c1}{\PYZsh{} DATA\PYZus{}PATH = \PYZsq{}https://raw.githubusercontent.com/richardfoltyn/python\PYZhy{}intro\PYZhy{}PGR/main/data\PYZsq{}}
\end{Verbatim}
\end{tcolorbox}

    \begin{tcolorbox}[breakable, size=fbox, boxrule=1pt, pad at break*=1mm,colback=cellbackground, colframe=cellborder]
\prompt{In}{incolor}{2}{\boxspacing}
\begin{Verbatim}[commandchars=\\\{\}]
\PY{k+kn}{import} \PY{n+nn}{numpy} \PY{k}{as} \PY{n+nn}{np}

\PY{c+c1}{\PYZsh{} Path to CSV file}
\PY{n}{file} \PY{o}{=} \PY{l+s+sa}{f}\PY{l+s+s1}{\PYZsq{}}\PY{l+s+si}{\PYZob{}}\PY{n}{DATA\PYZus{}PATH}\PY{l+s+si}{\PYZcb{}}\PY{l+s+s1}{/FRED.csv}\PY{l+s+s1}{\PYZsq{}}

\PY{c+c1}{\PYZsh{} load CSV}
\PY{n}{data} \PY{o}{=} \PY{n}{np}\PY{o}{.}\PY{n}{loadtxt}\PY{p}{(}\PY{n}{file}\PY{p}{,} \PY{n}{skiprows}\PY{o}{=}\PY{l+m+mi}{1}\PY{p}{,} \PY{n}{delimiter}\PY{o}{=}\PY{l+s+s1}{\PYZsq{}}\PY{l+s+s1}{,}\PY{l+s+s1}{\PYZsq{}}\PY{p}{)}

\PY{n}{data}\PY{p}{[}\PY{p}{:}\PY{l+m+mi}{2}\PY{p}{]}        \PY{c+c1}{\PYZsh{} Display first two rows}
\end{Verbatim}
\end{tcolorbox}

            \begin{tcolorbox}[breakable, size=fbox, boxrule=.5pt, pad at break*=1mm, opacityfill=0]
\prompt{Out}{outcolor}{2}{\boxspacing}
\begin{Verbatim}[commandchars=\\\{\}]
array([[1948. , 2118.5,   24. ,    3.8],
       [1949. , 2106.6,   23.8,    6. ]])
\end{Verbatim}
\end{tcolorbox}
        
    The default settings will in many cases be appropriate to load whatever
CSV file we might have. However, we'll occasionally want to specify the
following arguments to override the defaults:

\begin{itemize}
\tightlist
\item
  \texttt{delimiter}: Character used to separate individual fields
  (default: space).
\item
  \texttt{skiprows=n}: Skip the first \texttt{n} rows. For example, if
  the CSV file contains a header with variable names,
  \texttt{skiprows=1} needs to be specified as NumPy by default cannot
  process these names.
\item
  \texttt{dtype}: Enforce a particular data type for the resulting
  array.
\item
  \texttt{encoding}: Set the character encoding of the input data. This
  is usually not needed, but can be required to import data with
  non-latin characters that are not encoded using Unicode.
\end{itemize}

While \texttt{loadtxt()} is simple to use, it quickly reaches its limits
with more complex data sets. For example, when we try to load our sample
of universities with \texttt{loadtxt()}, we get the following error:

    \begin{tcolorbox}[breakable, size=fbox, boxrule=1pt, pad at break*=1mm,colback=cellbackground, colframe=cellborder]
\prompt{In}{incolor}{3}{\boxspacing}
\begin{Verbatim}[commandchars=\\\{\}]
\PY{k+kn}{import} \PY{n+nn}{numpy} \PY{k}{as} \PY{n+nn}{np}

\PY{n}{file} \PY{o}{=} \PY{l+s+sa}{f}\PY{l+s+s1}{\PYZsq{}}\PY{l+s+si}{\PYZob{}}\PY{n}{DATA\PYZus{}PATH}\PY{l+s+si}{\PYZcb{}}\PY{l+s+s1}{/universities.csv}\PY{l+s+s1}{\PYZsq{}}

\PY{c+c1}{\PYZsh{} Try to load CSV data that contains strings}
\PY{c+c1}{\PYZsh{} This will result in an error!}
\PY{n}{data} \PY{o}{=} \PY{n}{np}\PY{o}{.}\PY{n}{loadtxt}\PY{p}{(}\PY{n}{file}\PY{p}{,} \PY{n}{delimiter}\PY{o}{=}\PY{l+s+s1}{\PYZsq{}}\PY{l+s+s1}{;}\PY{l+s+s1}{\PYZsq{}}\PY{p}{,} \PY{n}{skiprows}\PY{o}{=}\PY{l+m+mi}{1}\PY{p}{)}
\end{Verbatim}
\end{tcolorbox}

    \begin{Verbatim}[commandchars=\\\{\}, frame=single, framerule=2mm, rulecolor=\color{outerrorbackground}]
\textcolor{ansi-red}{ValueError}\textcolor{ansi-red}{:} could not convert string to float: '"University of Glasgow"'


The above exception was the direct cause of the following exception:

\textcolor{ansi-red}{ValueError}\textcolor{ansi-red}{:} could not convert string '"University of Glasgow"' to float64 at row 0, column 1.

    \end{Verbatim}

    This code fails for two reasons:

\begin{enumerate}
\def\labelenumi{\arabic{enumi}.}
\tightlist
\item
  The file contains strings and floats, and \texttt{loadtxt()} by
  default cannot load mixed data (\eg strings and numerical data).
\item
  There are missing values (empty fields), which \texttt{loadtxt()}
  cannot handle either.
\end{enumerate}

The simplest way to address these issues is to use pandas to load the
data which we turn to in the next section.

    \hypertarget{saving-data-to-text-files}{%
\subsubsection{Saving data to text
files}\label{saving-data-to-text-files}}

To save a NumPy array to a CSV file, there is a logical counterpart to
\texttt{np.loadtxt()} which is called
\href{https://numpy.org/doc/stable/reference/generated/numpy.savetxt.html}{\texttt{np.savetxt()}}.

    \begin{tcolorbox}[breakable, size=fbox, boxrule=1pt, pad at break*=1mm,colback=cellbackground, colframe=cellborder]
\prompt{In}{incolor}{4}{\boxspacing}
\begin{Verbatim}[commandchars=\\\{\}]
\PY{k+kn}{import} \PY{n+nn}{numpy} \PY{k}{as} \PY{n+nn}{np}
\PY{k+kn}{import} \PY{n+nn}{os}\PY{n+nn}{.}\PY{n+nn}{path}
\PY{k+kn}{import} \PY{n+nn}{tempfile}

\PY{c+c1}{\PYZsh{} Generate three columns of 5 observations each}
\PY{n}{data} \PY{o}{=} \PY{n}{np}\PY{o}{.}\PY{n}{linspace}\PY{p}{(}\PY{l+m+mf}{0.0}\PY{p}{,} \PY{l+m+mf}{1.0}\PY{p}{,} \PY{l+m+mi}{15}\PY{p}{)}\PY{o}{.}\PY{n}{reshape}\PY{p}{(}\PY{p}{(}\PY{l+m+mi}{3}\PY{p}{,} \PY{l+m+mi}{5}\PY{p}{)}\PY{p}{)}

\PY{c+c1}{\PYZsh{} create temporary directory}
\PY{n}{d} \PY{o}{=} \PY{n}{tempfile}\PY{o}{.}\PY{n}{TemporaryDirectory}\PY{p}{(}\PY{p}{)}

\PY{c+c1}{\PYZsh{} path to CSV file}
\PY{n}{file} \PY{o}{=} \PY{n}{os}\PY{o}{.}\PY{n}{path}\PY{o}{.}\PY{n}{join}\PY{p}{(}\PY{n}{d}\PY{o}{.}\PY{n}{name}\PY{p}{,} \PY{l+s+s1}{\PYZsq{}}\PY{l+s+s1}{data.csv}\PY{l+s+s1}{\PYZsq{}}\PY{p}{)}

\PY{c+c1}{\PYZsh{} Print destination file \PYZhy{} this will be different each time}
\PY{n+nb}{print}\PY{p}{(}\PY{l+s+sa}{f}\PY{l+s+s1}{\PYZsq{}}\PY{l+s+s1}{Saving CSV file to }\PY{l+s+si}{\PYZob{}}\PY{n}{file}\PY{l+s+si}{\PYZcb{}}\PY{l+s+s1}{\PYZsq{}}\PY{p}{)}

\PY{c+c1}{\PYZsh{} Write NumPy array to CSV file. The fmt argument specifies}
\PY{c+c1}{\PYZsh{} that data should be saved as floating\PYZhy{}point using a}
\PY{c+c1}{\PYZsh{} field width of 8 characters and 5 decimal digits.}
\PY{n}{np}\PY{o}{.}\PY{n}{savetxt}\PY{p}{(}\PY{n}{file}\PY{p}{,} \PY{n}{data}\PY{p}{,} \PY{n}{delimiter}\PY{o}{=}\PY{l+s+s1}{\PYZsq{}}\PY{l+s+s1}{;}\PY{l+s+s1}{\PYZsq{}}\PY{p}{,} \PY{n}{fmt}\PY{o}{=}\PY{l+s+s1}{\PYZsq{}}\PY{l+s+si}{\PYZpc{}8.5f}\PY{l+s+s1}{\PYZsq{}}\PY{p}{)}
\end{Verbatim}
\end{tcolorbox}

    \begin{Verbatim}[commandchars=\\\{\}]
Saving CSV file to /tmp/tmp8hnv2edo/data.csv
    \end{Verbatim}

    The above code creates a \(5 \times 3\) matrix of floats and stores
these in the file \texttt{data.csv} using 5 significant digits.

We store the destination file in a temporary directory which we create
as follows:

\begin{itemize}
\tightlist
\item
  Because we cannot know in advance on which system this code is run
  (\eg the operating system and directory layout), we cannot hard-code
  a file path.
\item
  Moreover, we do not know whether the code is run with write
  permissions in any particular folder.
\item
  We work around this issue by asking the Python runtime to create a
  writeable temporary directory \emph{for the system where the code is
  being run}.
\item
  We use the routines in the
  \href{https://docs.python.org/3/library/tempfile.html}{\texttt{tempfile}}
  module to create this temporary directory.
\end{itemize}

Of course, on your own computer you do not need to use a temporary
directory, but can instead use any directory where your user has write
permissions. For example, on Windows you could use something along the
lines of

\begin{Shaded}
\begin{Highlighting}[]
\BuiltInTok{file} \OperatorTok{=} \StringTok{\textquotesingle{}C:/Users/Path/to/file.txt\textquotesingle{}}
\NormalTok{np.savetxt(}\BuiltInTok{file}\NormalTok{, data, delimiter}\OperatorTok{=}\StringTok{\textquotesingle{};\textquotesingle{}}\NormalTok{, fmt}\OperatorTok{=}\StringTok{\textquotesingle{}}\SpecialCharTok{\%8.5f}\StringTok{\textquotesingle{}}\NormalTok{)}
\end{Highlighting}
\end{Shaded}

You can even use relative paths. To store a file in the current working
directory it is sufficient to just pass the file name:

\begin{Shaded}
\begin{Highlighting}[]
\BuiltInTok{file} \OperatorTok{=} \StringTok{\textquotesingle{}file.txt\textquotesingle{}}
\NormalTok{np.savetxt(}\BuiltInTok{file}\NormalTok{, data, delimiter}\OperatorTok{=}\StringTok{\textquotesingle{};\textquotesingle{}}\NormalTok{, fmt}\OperatorTok{=}\StringTok{\textquotesingle{}}\SpecialCharTok{\%8.5f}\StringTok{\textquotesingle{}}\NormalTok{)}
\end{Highlighting}
\end{Shaded}


\hypertarget{inputoutput-with-pandas}{%
\subsection{Input/output with pandas}\label{inputoutput-with-pandas}}

Pandas's I/O routines are more powerful than those implemented in NumPy:

\begin{itemize}
\tightlist
\item
  They support reading and writing numerous file formats.
\item
  They support heterogeneous data without having to specify the data
  type in advance.
\item
  They gracefully handle missing values.
\end{itemize}

For these reasons, it is often preferable to directly use pandas to
process data instead of NumPy.

The most important routines are:

\begin{itemize}
\tightlist
\item
  \href{https://pandas.pydata.org/pandas-docs/stable/reference/api/pandas.read_csv.html}{\texttt{read\_csv()}},
  \href{https://pandas.pydata.org/pandas-docs/stable/reference/api/pandas.DataFrame.to_csv.html}{\texttt{to\_csv()}}:
  Read or write CSV text files
\item
  \href{https://pandas.pydata.org/pandas-docs/stable/reference/api/pandas.read_fwf.html}{\texttt{read\_fwf()}}:
  Read data with fixed field widths, \ie text data that does not use
  delimiters to separate fields.
\item
  \href{https://pandas.pydata.org/pandas-docs/stable/reference/api/pandas.read_excel.html}{\texttt{read\_excel()}},
  \href{https://pandas.pydata.org/pandas-docs/stable/reference/api/pandas.DataFrame.to_excel.html}{\texttt{to\_excel()}}:
  Read or write Excel spreadsheets
\item
  \href{https://pandas.pydata.org/pandas-docs/stable/reference/api/pandas.read_stata.html}{\texttt{read\_stata()}},
  \href{https://pandas.pydata.org/pandas-docs/stable/reference/api/pandas.DataFrame.to_stata.html}{\texttt{to\_stata()}}:
  Read or write Stata's \texttt{.dta} files.
\item
  \href{https://pandas.pydata.org/docs/reference/api/pandas.read_pickle.html}{\texttt{read\_pickle()}},
  \href{https://pandas.pydata.org/docs/reference/api/pandas.DataFrame.to_pickle.html\#pandas.DataFrame.to_pickle}{\texttt{to\_pickle()}}:
  Read or write Python's binary pickle format, optionally using
  compression (see
  \href{https://pandas.pydata.org/pandas-docs/stable/user_guide/io.html\#io-pickle}{here}
  for details). This should only be used to create temporary files, not
  to store data permanently.
\end{itemize}

For a complete list of I/O routines, see the
\href{https://pandas.pydata.org/pandas-docs/stable/user_guide/io.html}{official
documentation}.

To illustrate, we repeat the above examples using pandas's
\href{https://pandas.pydata.org/pandas-docs/stable/reference/api/pandas.read_csv.html}{\texttt{read\_csv()}}.
Since the FRED data contains only floating-point data, the result is
very similar to reading in a NumPy array.

    \begin{tcolorbox}[breakable, size=fbox, boxrule=1pt, pad at break*=1mm,colback=cellbackground, colframe=cellborder]
\prompt{In}{incolor}{5}{\boxspacing}
\begin{Verbatim}[commandchars=\\\{\}]
\PY{k+kn}{import} \PY{n+nn}{pandas} \PY{k}{as} \PY{n+nn}{pd}

\PY{c+c1}{\PYZsh{} relative path to CSV file}
\PY{n}{file} \PY{o}{=} \PY{l+s+sa}{f}\PY{l+s+s1}{\PYZsq{}}\PY{l+s+si}{\PYZob{}}\PY{n}{DATA\PYZus{}PATH}\PY{l+s+si}{\PYZcb{}}\PY{l+s+s1}{/FRED.csv}\PY{l+s+s1}{\PYZsq{}}

\PY{n}{df} \PY{o}{=} \PY{n}{pd}\PY{o}{.}\PY{n}{read\PYZus{}csv}\PY{p}{(}\PY{n}{file}\PY{p}{,} \PY{n}{sep}\PY{o}{=}\PY{l+s+s1}{\PYZsq{}}\PY{l+s+s1}{,}\PY{l+s+s1}{\PYZsq{}}\PY{p}{)}
\PY{n}{df}\PY{o}{.}\PY{n}{head}\PY{p}{(}\PY{l+m+mi}{2}\PY{p}{)}          \PY{c+c1}{\PYZsh{} Display the first 2 rows of data}
\end{Verbatim}
\end{tcolorbox}

            \begin{tcolorbox}[breakable, size=fbox, boxrule=.5pt, pad at break*=1mm, opacityfill=0]
\prompt{Out}{outcolor}{5}{\boxspacing}
\begin{Verbatim}[commandchars=\\\{\}]
   Year     GDP   CPI  UNRATE
0  1948  2118.5  24.0     3.8
1  1949  2106.6  23.8     6.0
\end{Verbatim}
\end{tcolorbox}
        
    The difference between NumPy and pandas become obvious when we try to
load our university data: this works out of the box:

    \begin{tcolorbox}[breakable, size=fbox, boxrule=1pt, pad at break*=1mm,colback=cellbackground, colframe=cellborder]
\prompt{In}{incolor}{6}{\boxspacing}
\begin{Verbatim}[commandchars=\\\{\}]
\PY{k+kn}{import} \PY{n+nn}{pandas} \PY{k}{as} \PY{n+nn}{pd}

\PY{c+c1}{\PYZsh{} relative path to CSV file}
\PY{n}{file} \PY{o}{=} \PY{l+s+sa}{f}\PY{l+s+s1}{\PYZsq{}}\PY{l+s+si}{\PYZob{}}\PY{n}{DATA\PYZus{}PATH}\PY{l+s+si}{\PYZcb{}}\PY{l+s+s1}{/universities.csv}\PY{l+s+s1}{\PYZsq{}}

\PY{n}{df} \PY{o}{=} \PY{n}{pd}\PY{o}{.}\PY{n}{read\PYZus{}csv}\PY{p}{(}\PY{n}{file}\PY{p}{,} \PY{n}{sep}\PY{o}{=}\PY{l+s+s1}{\PYZsq{}}\PY{l+s+s1}{;}\PY{l+s+s1}{\PYZsq{}}\PY{p}{)}
\PY{n}{df}\PY{o}{.}\PY{n}{tail}\PY{p}{(}\PY{l+m+mi}{3}\PY{p}{)}      \PY{c+c1}{\PYZsh{} show last 3 rows}
\end{Verbatim}
\end{tcolorbox}

            \begin{tcolorbox}[breakable, size=fbox, boxrule=.5pt, pad at break*=1mm, opacityfill=0]
\prompt{Out}{outcolor}{6}{\boxspacing}
\begin{Verbatim}[commandchars=\\\{\}]
                   Institution           Country  Founded  Students   Staff  \textbackslash{}
20      University of Stirling          Scotland     1967      9548     NaN
21  Queen’s University Belfast  Northern Ireland     1810     18438  2414.0
22          Swansea University             Wales     1920     20620     NaN

     Admin  Budget  Russell
20  1872.0   113.3        0
21  1489.0   369.2        1
22  3290.0     NaN        0
\end{Verbatim}
\end{tcolorbox}
        
    Note that missing values are correctly converted to
\href{https://numpy.org/doc/stable/reference/constants.html\#numpy.NAN}{\texttt{np.nan}}.

Unlike NumPy, pandas can also process other popular data formats such as
MS Excel files (or OpenDocument spreadsheets):

    \begin{tcolorbox}[breakable, size=fbox, boxrule=1pt, pad at break*=1mm,colback=cellbackground, colframe=cellborder]
\prompt{In}{incolor}{7}{\boxspacing}
\begin{Verbatim}[commandchars=\\\{\}]
\PY{k+kn}{import} \PY{n+nn}{pandas} \PY{k}{as} \PY{n+nn}{pd}

\PY{c+c1}{\PYZsh{} Excel file containing university data}
\PY{n}{file} \PY{o}{=} \PY{l+s+sa}{f}\PY{l+s+s1}{\PYZsq{}}\PY{l+s+si}{\PYZob{}}\PY{n}{DATA\PYZus{}PATH}\PY{l+s+si}{\PYZcb{}}\PY{l+s+s1}{/universities.xlsx}\PY{l+s+s1}{\PYZsq{}}

\PY{n}{df} \PY{o}{=} \PY{n}{pd}\PY{o}{.}\PY{n}{read\PYZus{}excel}\PY{p}{(}\PY{n}{file}\PY{p}{,} \PY{n}{sheet\PYZus{}name}\PY{o}{=}\PY{l+s+s1}{\PYZsq{}}\PY{l+s+s1}{universities}\PY{l+s+s1}{\PYZsq{}}\PY{p}{)}
\PY{n}{df}\PY{o}{.}\PY{n}{head}\PY{p}{(}\PY{l+m+mi}{3}\PY{p}{)}
\end{Verbatim}
\end{tcolorbox}

            \begin{tcolorbox}[breakable, size=fbox, boxrule=.5pt, pad at break*=1mm, opacityfill=0]
\prompt{Out}{outcolor}{7}{\boxspacing}
\begin{Verbatim}[commandchars=\\\{\}]
                Institution   Country  Founded  Students   Staff   Admin  \textbackslash{}
0     University of Glasgow  Scotland     1451     30805  2942.0  4003.0
1   University of Edinburgh  Scotland     1583     34275  4589.0  6107.0
2  University of St Andrews  Scotland     1413      8984  1137.0  1576.0

   Budget  Russell
0   626.5        1
1  1102.0        1
2   251.2        0
\end{Verbatim}
\end{tcolorbox}
        
    The routine \texttt{read\_excel()} takes the argument
\texttt{sheet\_name} to specify the sheet that should be read.

\begin{itemize}
\tightlist
\item
  Note that the Python package
  \href{https://pypi.org/project/openpyxl/}{\texttt{openpyxl}} needs to
  be installed in order to read files from Excel 2003 and above.
\item
  To read older Excel files (\texttt{.xls}), you need the package
  \href{https://pypi.org/project/xlrd/}{\texttt{xlrd}}.
\end{itemize}

Finally, we often encounter text files with fixed field widths, since
this is a commonly used format in older applications (for example,
fixed-width files are easy to create in Fortran). To illustrate, the
fixed-width variant of our FRED data looks like this:

\begin{verbatim}
 Year GDP    CPI  UNRATE
 1948 2118.5   24     3.8
 1949 2106.6 23.8       6
 1950 2289.5 24.1     5.2
 1951 2473.8   26     3.3
 1952 2574.9 26.6       3
\end{verbatim}

You see that the column \texttt{Year} occupies the first 5 characters,
the \texttt{GDP} column the next 7 characters, and so on. To read such
files, the width (\ie the number of characters) has to be explicitly
specified:

    \begin{tcolorbox}[breakable, size=fbox, boxrule=1pt, pad at break*=1mm,colback=cellbackground, colframe=cellborder]
\prompt{In}{incolor}{8}{\boxspacing}
\begin{Verbatim}[commandchars=\\\{\}]
\PY{k+kn}{import} \PY{n+nn}{pandas} \PY{k}{as} \PY{n+nn}{pd}

\PY{c+c1}{\PYZsh{} File name of FRED data, stored as fixed\PYZhy{}width text}
\PY{n}{file} \PY{o}{=} \PY{l+s+sa}{f}\PY{l+s+s1}{\PYZsq{}}\PY{l+s+si}{\PYZob{}}\PY{n}{DATA\PYZus{}PATH}\PY{l+s+si}{\PYZcb{}}\PY{l+s+s1}{/FRED\PYZhy{}fixed.csv}\PY{l+s+s1}{\PYZsq{}}

\PY{c+c1}{\PYZsh{} field widths are passed as list to read\PYZus{}fwf()}
\PY{n}{df} \PY{o}{=} \PY{n}{pd}\PY{o}{.}\PY{n}{read\PYZus{}fwf}\PY{p}{(}\PY{n}{file}\PY{p}{,} \PY{n}{widths}\PY{o}{=}\PY{p}{[}\PY{l+m+mi}{5}\PY{p}{,} \PY{l+m+mi}{7}\PY{p}{,} \PY{l+m+mi}{5}\PY{p}{,} \PY{l+m+mi}{8}\PY{p}{]}\PY{p}{)}
\PY{n}{df}\PY{o}{.}\PY{n}{head}\PY{p}{(}\PY{l+m+mi}{3}\PY{p}{)}
\end{Verbatim}
\end{tcolorbox}

            \begin{tcolorbox}[breakable, size=fbox, boxrule=.5pt, pad at break*=1mm, opacityfill=0]
\prompt{Out}{outcolor}{8}{\boxspacing}
\begin{Verbatim}[commandchars=\\\{\}]
   Year     GDP   CPI  UNRATE
0  1948  2118.5  24.0     3.8
1  1949  2106.6  23.8     6.0
2  1950  2289.5  24.1     5.2
\end{Verbatim}
\end{tcolorbox}
        
    Here the \texttt{widths} argument accepts a list that contains the
number of characters to be used for each field.


\hypertarget{retrieving-macroeconomic-financial-data-from-the-web}{%
\subsection{Retrieving macroeconomic / financial data from the
web}\label{retrieving-macroeconomic-financial-data-from-the-web}}

    \hypertarget{yahoo-finance-data}{%
\subsubsection{Yahoo! Finance data}\label{yahoo-finance-data}}

\href{https://pypi.org/project/yfinance/}{\texttt{yfinance}} is a
user-written library to access data from
\href{https://finance.yahoo.com/}{Yahoo! Finance} using the public API
(see the project's \href{https://github.com/ranaroussi/yfinance}{GitHub
repository} for detailed examples). This project is not affiliated with
Yahoo! Finance and is intended for personal use only. Before using the
library, it needs to be installed from PyPi as follows:

\begin{Shaded}
\begin{Highlighting}[]
\ExtensionTok{pip}\NormalTok{ install yfinance }
\end{Highlighting}
\end{Shaded}

    \begin{tcolorbox}[breakable, size=fbox, boxrule=1pt, pad at break*=1mm,colback=cellbackground, colframe=cellborder]
\prompt{In}{incolor}{9}{\boxspacing}
\begin{Verbatim}[commandchars=\\\{\}]
\PY{c+c1}{\PYZsh{} When running via Google Colab, uncomment and execute the following line}
\PY{c+c1}{\PYZsh{}! pip install yfinance}
\end{Verbatim}
\end{tcolorbox}

    \texttt{yfinance} allows us to retrieve information for a single symbol
via properties of the \texttt{Ticker} object, or for multiple ticker
symbols at once.

\vspace{1em}\emph{Example: Retrieving data for a single symbol}

We first use the API to retrieve data for a single symbol, in this case
the \href{https://finance.yahoo.com/quote/\%5EGSPC}{S\&P 500 index}
which has the (somewhat unusual) ticker symbol \texttt{\^{}GSPS}. One
can easily find the desired ticker symbol by searching for some stock,
index, currency or other asset on Yahoo! Finance.

    \begin{tcolorbox}[breakable, size=fbox, boxrule=1pt, pad at break*=1mm,colback=cellbackground, colframe=cellborder]
\prompt{In}{incolor}{10}{\boxspacing}
\begin{Verbatim}[commandchars=\\\{\}]
\PY{k+kn}{import} \PY{n+nn}{yfinance} \PY{k}{as} \PY{n+nn}{yf}

\PY{c+c1}{\PYZsh{} Symbol for S\PYZam{}P 500 index}
\PY{n}{symbol} \PY{o}{=} \PY{l+s+s1}{\PYZsq{}}\PY{l+s+s1}{\PYZca{}GSPC}\PY{l+s+s1}{\PYZsq{}}

\PY{c+c1}{\PYZsh{} Create ticker object}
\PY{n}{ticker} \PY{o}{=} \PY{n}{yf}\PY{o}{.}\PY{n}{Ticker}\PY{p}{(}\PY{n}{symbol}\PY{p}{)}
\end{Verbatim}
\end{tcolorbox}

    We can now use the attributes of the \texttt{ticker} object to get all
sorts of information. For example, we can get some meta data from the
\texttt{info} attribute as follows:

    \begin{tcolorbox}[breakable, size=fbox, boxrule=1pt, pad at break*=1mm,colback=cellbackground, colframe=cellborder]
\prompt{In}{incolor}{11}{\boxspacing}
\begin{Verbatim}[commandchars=\\\{\}]
\PY{c+c1}{\PYZsh{} Descriptive name and asset class}
\PY{n}{shortname} \PY{o}{=} \PY{n}{ticker}\PY{o}{.}\PY{n}{info}\PY{p}{[}\PY{l+s+s1}{\PYZsq{}}\PY{l+s+s1}{shortName}\PY{l+s+s1}{\PYZsq{}}\PY{p}{]}
\PY{n}{quoteType} \PY{o}{=} \PY{n}{ticker}\PY{o}{.}\PY{n}{info}\PY{p}{[}\PY{l+s+s1}{\PYZsq{}}\PY{l+s+s1}{quoteType}\PY{l+s+s1}{\PYZsq{}}\PY{p}{]}

\PY{c+c1}{\PYZsh{} 52\PYZhy{}week low and high}
\PY{n}{low} \PY{o}{=} \PY{n}{ticker}\PY{o}{.}\PY{n}{info}\PY{p}{[}\PY{l+s+s1}{\PYZsq{}}\PY{l+s+s1}{fiftyTwoWeekLow}\PY{l+s+s1}{\PYZsq{}}\PY{p}{]}
\PY{n}{high} \PY{o}{=} \PY{n}{ticker}\PY{o}{.}\PY{n}{info}\PY{p}{[}\PY{l+s+s1}{\PYZsq{}}\PY{l+s+s1}{fiftyTwoWeekHigh}\PY{l+s+s1}{\PYZsq{}}\PY{p}{]}

\PY{n+nb}{print}\PY{p}{(}\PY{l+s+sa}{f}\PY{l+s+s1}{\PYZsq{}}\PY{l+s+si}{\PYZob{}}\PY{n}{shortname}\PY{l+s+si}{\PYZcb{}}\PY{l+s+s1}{ is an }\PY{l+s+si}{\PYZob{}}\PY{n}{quoteType}\PY{l+s+si}{\PYZcb{}}\PY{l+s+s1}{\PYZsq{}}\PY{p}{)}
\PY{n+nb}{print}\PY{p}{(}\PY{l+s+sa}{f}\PY{l+s+s1}{\PYZsq{}}\PY{l+s+si}{\PYZob{}}\PY{n}{shortname}\PY{l+s+si}{\PYZcb{}}\PY{l+s+s1}{ 52\PYZhy{}week range: }\PY{l+s+si}{\PYZob{}}\PY{n}{low}\PY{l+s+si}{\PYZcb{}}\PY{l+s+s1}{ \PYZhy{} }\PY{l+s+si}{\PYZob{}}\PY{n}{high}\PY{l+s+si}{\PYZcb{}}\PY{l+s+s1}{\PYZsq{}}\PY{p}{)}

\PY{c+c1}{\PYZsh{} To see which keys are available, use the keys() method}
\PY{c+c1}{\PYZsh{} ticker.info.keys()}
\end{Verbatim}
\end{tcolorbox}

    \begin{Verbatim}[commandchars=\\\{\}]
S\&P 500 is an INDEX
S\&P 500 52-week range: 3491.58 - 4325.28
    \end{Verbatim}

    We use the \texttt{history} attribute to get detailed price data. Unless
we want all available data, we should select the relevant period using
the \texttt{start=...} and \texttt{end=...} arguments.

    \begin{tcolorbox}[breakable, size=fbox, boxrule=1pt, pad at break*=1mm,colback=cellbackground, colframe=cellborder]
\prompt{In}{incolor}{12}{\boxspacing}
\begin{Verbatim}[commandchars=\\\{\}]
\PY{c+c1}{\PYZsh{} Retrieve daily index values data for first quarter of this year}
\PY{n}{daily} \PY{o}{=} \PY{n}{ticker}\PY{o}{.}\PY{n}{history}\PY{p}{(}\PY{n}{start}\PY{o}{=}\PY{l+s+s1}{\PYZsq{}}\PY{l+s+s1}{2023\PYZhy{}01\PYZhy{}01}\PY{l+s+s1}{\PYZsq{}}\PY{p}{,} \PY{n}{end}\PY{o}{=}\PY{l+s+s1}{\PYZsq{}}\PY{l+s+s1}{2023\PYZhy{}03\PYZhy{}31}\PY{l+s+s1}{\PYZsq{}}\PY{p}{)}

\PY{c+c1}{\PYZsh{} Print first 5 rows}
\PY{n}{daily}\PY{o}{.}\PY{n}{head}\PY{p}{(}\PY{p}{)}
\end{Verbatim}
\end{tcolorbox}

            \begin{tcolorbox}[breakable, size=fbox, boxrule=.5pt, pad at break*=1mm, opacityfill=0]
\prompt{Out}{outcolor}{12}{\boxspacing}
\begin{Verbatim}[commandchars=\\\{\}]
                                  Open         High          Low        Close  \textbackslash{}
Date
2023-01-03 00:00:00-05:00  3853.290039  3878.459961  3794.330078  3824.139893
2023-01-04 00:00:00-05:00  3840.360107  3873.159912  3815.770020  3852.969971
2023-01-05 00:00:00-05:00  3839.739990  3839.739990  3802.419922  3808.100098
2023-01-06 00:00:00-05:00  3823.370117  3906.189941  3809.560059  3895.080078
2023-01-09 00:00:00-05:00  3910.820068  3950.570068  3890.419922  3892.090088

                               Volume  Dividends  Stock Splits
Date
2023-01-03 00:00:00-05:00  3959140000        0.0           0.0
2023-01-04 00:00:00-05:00  4414080000        0.0           0.0
2023-01-05 00:00:00-05:00  3893450000        0.0           0.0
2023-01-06 00:00:00-05:00  3923560000        0.0           0.0
2023-01-09 00:00:00-05:00  4311770000        0.0           0.0
\end{Verbatim}
\end{tcolorbox}
        
    We can then use this data to plot the daily closing price and trading
volume.

    \begin{tcolorbox}[breakable, size=fbox, boxrule=1pt, pad at break*=1mm,colback=cellbackground, colframe=cellborder]
\prompt{In}{incolor}{13}{\boxspacing}
\begin{Verbatim}[commandchars=\\\{\}]
\PY{k+kn}{import} \PY{n+nn}{matplotlib}\PY{n+nn}{.}\PY{n+nn}{pyplot} \PY{k}{as} \PY{n+nn}{plt}

\PY{n}{fix}\PY{p}{,} \PY{n}{ax} \PY{o}{=} \PY{n}{plt}\PY{o}{.}\PY{n}{subplots}\PY{p}{(}\PY{l+m+mi}{1}\PY{p}{,} \PY{l+m+mi}{1}\PY{p}{,} \PY{n}{figsize}\PY{o}{=}\PY{p}{(}\PY{l+m+mi}{7}\PY{p}{,}\PY{l+m+mf}{3.5}\PY{p}{)}\PY{p}{)}

\PY{c+c1}{\PYZsh{} Plot closing price}
\PY{n}{ax}\PY{o}{.}\PY{n}{plot}\PY{p}{(}\PY{n}{daily}\PY{o}{.}\PY{n}{index}\PY{p}{,} \PY{n}{daily}\PY{p}{[}\PY{l+s+s1}{\PYZsq{}}\PY{l+s+s1}{Close}\PY{l+s+s1}{\PYZsq{}}\PY{p}{]}\PY{p}{,} \PY{n}{color}\PY{o}{=}\PY{l+s+s1}{\PYZsq{}}\PY{l+s+s1}{darkblue}\PY{l+s+s1}{\PYZsq{}}\PY{p}{,} \PY{n}{marker}\PY{o}{=}\PY{l+s+s1}{\PYZsq{}}\PY{l+s+s1}{o}\PY{l+s+s1}{\PYZsq{}}\PY{p}{,} \PY{n}{ms}\PY{o}{=}\PY{l+m+mi}{3}\PY{p}{,} \PY{n}{lw}\PY{o}{=}\PY{l+m+mi}{1}\PY{p}{)}
\PY{n}{ax}\PY{o}{.}\PY{n}{set\PYZus{}ylabel}\PY{p}{(}\PY{l+s+s1}{\PYZsq{}}\PY{l+s+s1}{Price at close}\PY{l+s+s1}{\PYZsq{}}\PY{p}{)}
\PY{n}{ax}\PY{o}{.}\PY{n}{legend}\PY{p}{(}\PY{p}{[}\PY{l+s+s1}{\PYZsq{}}\PY{l+s+s1}{Price at close}\PY{l+s+s1}{\PYZsq{}}\PY{p}{]}\PY{p}{,} \PY{n}{loc}\PY{o}{=}\PY{l+s+s1}{\PYZsq{}}\PY{l+s+s1}{upper right}\PY{l+s+s1}{\PYZsq{}}\PY{p}{)}

\PY{c+c1}{\PYZsh{} Create secondary y\PYZhy{}axis for trading volume}
\PY{n}{ax2} \PY{o}{=} \PY{n}{ax}\PY{o}{.}\PY{n}{twinx}\PY{p}{(}\PY{p}{)}

\PY{c+c1}{\PYZsh{} Plot trading volume as bar chart}
\PY{n}{ax2}\PY{o}{.}\PY{n}{bar}\PY{p}{(}\PY{n}{daily}\PY{o}{.}\PY{n}{index}\PY{p}{,} \PY{n}{daily}\PY{p}{[}\PY{l+s+s1}{\PYZsq{}}\PY{l+s+s1}{Volume}\PY{l+s+s1}{\PYZsq{}}\PY{p}{]}\PY{p}{,} \PY{n}{color}\PY{o}{=}\PY{l+s+s1}{\PYZsq{}}\PY{l+s+s1}{\PYZsh{}666666}\PY{l+s+s1}{\PYZsq{}}\PY{p}{,} \PY{n}{alpha}\PY{o}{=}\PY{l+m+mf}{0.4}\PY{p}{,} \PY{n}{zorder}\PY{o}{=}\PY{o}{\PYZhy{}}\PY{l+m+mi}{1}\PY{p}{,} \PY{n}{lw}\PY{o}{=}\PY{l+m+mi}{0}\PY{p}{)}
\PY{n}{ax2}\PY{o}{.}\PY{n}{set\PYZus{}ylim}\PY{p}{(}\PY{p}{(}\PY{l+m+mf}{0.0}\PY{p}{,} \PY{l+m+mf}{1.0e10}\PY{p}{)}\PY{p}{)}
\PY{n}{ax2}\PY{o}{.}\PY{n}{set\PYZus{}ylabel}\PY{p}{(}\PY{l+s+s1}{\PYZsq{}}\PY{l+s+s1}{Volume}\PY{l+s+s1}{\PYZsq{}}\PY{p}{)}
\end{Verbatim}
\end{tcolorbox}

            \begin{tcolorbox}[breakable, size=fbox, boxrule=.5pt, pad at break*=1mm, opacityfill=0]
\prompt{Out}{outcolor}{13}{\boxspacing}
\begin{Verbatim}[commandchars=\\\{\}]
Text(0, 0.5, 'Volume')
\end{Verbatim}
\end{tcolorbox}
        
    \begin{center}
    \adjustimage{max size={0.9\linewidth}{0.9\paperheight}}{unit07_files/unit07_30_1.pdf}
    \end{center}
    
    The above code uses
\href{https://matplotlib.org/stable/api/_as_gen/matplotlib.axes.Axes.twinx.html}{\texttt{twinx()}}
to create a second (invisible) \(x\)-axis with an independent \(y\)-axis
which allows us to plot the trading volume on a different scale.

\vspace{1em}\emph{Example: Retrieving data for multiple symbols}

We can download trading data for multiple symbols at once using the
\texttt{download()} function. Unlike the \texttt{Ticker} class, this
immediately returns a \texttt{DataFrame} containing data similar to the
\texttt{history} method we called previously, but now the column index
contains an additional level for each ticker symbol. For example, to get
the trading data for Amazon and Microsoft for the first 3 months of
2023, we proceed as follows:

    \begin{tcolorbox}[breakable, size=fbox, boxrule=1pt, pad at break*=1mm,colback=cellbackground, colframe=cellborder]
\prompt{In}{incolor}{14}{\boxspacing}
\begin{Verbatim}[commandchars=\\\{\}]
\PY{k+kn}{import} \PY{n+nn}{yfinance} \PY{k}{as} \PY{n+nn}{yf}

\PY{c+c1}{\PYZsh{} Get data for Amazon (AMZN) and Microsoft (MSFT) for first quarter of 2023}
\PY{n}{data} \PY{o}{=} \PY{n}{yf}\PY{o}{.}\PY{n}{download}\PY{p}{(}\PY{p}{(}\PY{l+s+s1}{\PYZsq{}}\PY{l+s+s1}{AMZN}\PY{l+s+s1}{\PYZsq{}}\PY{p}{,} \PY{l+s+s1}{\PYZsq{}}\PY{l+s+s1}{MSFT}\PY{l+s+s1}{\PYZsq{}}\PY{p}{)}\PY{p}{,} \PY{n}{start}\PY{o}{=}\PY{l+s+s1}{\PYZsq{}}\PY{l+s+s1}{2023\PYZhy{}01\PYZhy{}01}\PY{l+s+s1}{\PYZsq{}}\PY{p}{,} \PY{n}{end}\PY{o}{=}\PY{l+s+s1}{\PYZsq{}}\PY{l+s+s1}{2023\PYZhy{}03\PYZhy{}31}\PY{l+s+s1}{\PYZsq{}}\PY{p}{)}
\PY{n}{data}\PY{o}{.}\PY{n}{head}\PY{p}{(}\PY{p}{)}
\end{Verbatim}
\end{tcolorbox}

    \begin{Verbatim}[commandchars=\\\{\}]
[*********************100\%***********************]  2 of 2 completed
    \end{Verbatim}

            \begin{tcolorbox}[breakable, size=fbox, boxrule=.5pt, pad at break*=1mm, opacityfill=0]
\prompt{Out}{outcolor}{14}{\boxspacing}
\begin{Verbatim}[commandchars=\\\{\}]
            Adj Close                  Close                   High  \textbackslash{}
                 AMZN        MSFT       AMZN        MSFT       AMZN
Date
2023-01-03  85.820000  238.460144  85.820000  239.580002  86.959999
2023-01-04  85.139999  228.029129  85.139999  229.100006  86.980003
2023-01-05  83.120003  221.270844  83.120003  222.309998  85.419998
2023-01-06  86.080002  223.878601  86.080002  224.929993  86.400002
2023-01-09  87.360001  226.058380  87.360001  227.119995  89.480003

                              Low                   Open              \textbackslash{}
                  MSFT       AMZN        MSFT       AMZN        MSFT
Date
2023-01-03  245.750000  84.209999  237.399994  85.459999  243.080002
2023-01-04  232.869995  83.360001  225.960007  86.550003  232.279999
2023-01-05  227.550003  83.070000  221.759995  85.330002  227.199997
2023-01-06  225.759995  81.430000  219.350006  83.029999  223.000000
2023-01-09  231.240005  87.080002  226.410004  87.459999  226.449997

              Volume
                AMZN      MSFT
Date
2023-01-03  76706000  25740000
2023-01-04  68885100  50623400
2023-01-05  67930800  39585600
2023-01-06  83303400  43613600
2023-01-09  65266100  27369800
\end{Verbatim}
\end{tcolorbox}
        
    To extract data for a particular symbol, we have to take into account
the hierarchical column index:

    \begin{tcolorbox}[breakable, size=fbox, boxrule=1pt, pad at break*=1mm,colback=cellbackground, colframe=cellborder]
\prompt{In}{incolor}{15}{\boxspacing}
\begin{Verbatim}[commandchars=\\\{\}]
\PY{c+c1}{\PYZsh{} Use hierarchical indexing to get data for Amazon}
\PY{n}{data}\PY{p}{[}\PY{p}{(}\PY{l+s+s1}{\PYZsq{}}\PY{l+s+s1}{Close}\PY{l+s+s1}{\PYZsq{}}\PY{p}{,} \PY{l+s+s1}{\PYZsq{}}\PY{l+s+s1}{AMZN}\PY{l+s+s1}{\PYZsq{}}\PY{p}{)}\PY{p}{]}\PY{o}{.}\PY{n}{head}\PY{p}{(}\PY{p}{)}
\end{Verbatim}
\end{tcolorbox}

            \begin{tcolorbox}[breakable, size=fbox, boxrule=.5pt, pad at break*=1mm, opacityfill=0]
\prompt{Out}{outcolor}{15}{\boxspacing}
\begin{Verbatim}[commandchars=\\\{\}]
Date
2023-01-03    85.820000
2023-01-04    85.139999
2023-01-05    83.120003
2023-01-06    86.080002
2023-01-09    87.360001
Name: (Close, AMZN), dtype: float64
\end{Verbatim}
\end{tcolorbox}
        
    \begin{tcolorbox}[breakable, size=fbox, boxrule=1pt, pad at break*=1mm,colback=cellbackground, colframe=cellborder]
\prompt{In}{incolor}{16}{\boxspacing}
\begin{Verbatim}[commandchars=\\\{\}]
\PY{c+c1}{\PYZsh{} Plot daily returns for both stocks}
\PY{n}{returns} \PY{o}{=} \PY{n}{data}\PY{p}{[}\PY{l+s+s1}{\PYZsq{}}\PY{l+s+s1}{Close}\PY{l+s+s1}{\PYZsq{}}\PY{p}{]}\PY{o}{.}\PY{n}{pct\PYZus{}change}\PY{p}{(}\PY{p}{)} \PY{o}{*} \PY{l+m+mf}{100.0}
\PY{n}{returns}\PY{o}{.}\PY{n}{plot}\PY{p}{(}\PY{n}{y}\PY{o}{=}\PY{p}{[}\PY{l+s+s1}{\PYZsq{}}\PY{l+s+s1}{AMZN}\PY{l+s+s1}{\PYZsq{}}\PY{p}{,} \PY{l+s+s1}{\PYZsq{}}\PY{l+s+s1}{MSFT}\PY{l+s+s1}{\PYZsq{}}\PY{p}{]}\PY{p}{,} \PY{n}{ylabel}\PY{o}{=}\PY{l+s+s1}{\PYZsq{}}\PY{l+s+s1}{Daily returns (}\PY{l+s+s1}{\PYZpc{}}\PY{l+s+s1}{)}\PY{l+s+s1}{\PYZsq{}}\PY{p}{)}
\end{Verbatim}
\end{tcolorbox}

            \begin{tcolorbox}[breakable, size=fbox, boxrule=.5pt, pad at break*=1mm, opacityfill=0]
\prompt{Out}{outcolor}{16}{\boxspacing}
\begin{Verbatim}[commandchars=\\\{\}]
<Axes: xlabel='Date', ylabel='Daily returns (\%)'>
\end{Verbatim}
\end{tcolorbox}
        
    \begin{center}
    \adjustimage{max size={0.9\linewidth}{0.9\paperheight}}{unit07_files/unit07_36_1.pdf}
    \end{center}
    

\hypertarget{pandas-datareader}{%
\subsubsection{Pandas Datareader}\label{pandas-datareader}}

\href{https://github.com/pydata/pandas-datareader}{\texttt{pandas-datareader}}
is a Python library that fetches online data from multiple sources and
returns them as pandas \texttt{DataFrame} objects. Despite its name,
this library is not included in \texttt{pandas} and may need to be
installed separately, \eg by running

\begin{Shaded}
\begin{Highlighting}[]
\ExtensionTok{pip}\NormalTok{ install pandas{-}datareader}
\end{Highlighting}
\end{Shaded}

The aim is to provide a uniform API to access data from multiple
sources, including those covered in other sections in this unit. See the
official
\href{https://pandas-datareader.readthedocs.io/en/latest/}{documentation}
for supported data sources and how to access them.

    \begin{tcolorbox}[breakable, size=fbox, boxrule=1pt, pad at break*=1mm,colback=cellbackground, colframe=cellborder]
\prompt{In}{incolor}{17}{\boxspacing}
\begin{Verbatim}[commandchars=\\\{\}]
\PY{c+c1}{\PYZsh{} Uncomment and execute the following line if running in Google Colab}
\PY{c+c1}{\PYZsh{} ! pip install pandas\PYZhy{}datareader}
\end{Verbatim}
\end{tcolorbox}

\vspace{1em}\emph{Example: Downloading data from FRED}

As a first illustration, we fetch macroeconomic data from
\href{https://fred.stlouisfed.org/}{FRED}, or Federal Reserve Economic
Data. FRED is provided by the Federal Reserve of St.~Louis and is one of
the most important macroeconomic online databases (at least for
US-centric data).

An alternative (but more complicated) way to access this data is via the
\texttt{fredapi} library, which we examine below. With
\texttt{pandas-datareader}, no API key is required to access FRED which
makes using it a little simpler than \texttt{fredapi}.

    In order to retrieve any data, we first need to identify the series
name. This is easiest done by searching for the data on
\href{https://fred.stlouisfed.org/}{FRED} using your browser and copying
the series name, highlighted in red in the screenshot below.

\begin{figure}
\centering
\includegraphics{FRED_series_name_GDP.png}
\caption{FRED series name}
\end{figure}

For example, if we want to retrieve the
\href{https://fred.stlouisfed.org/series/GDP}{US Gross Domestic Product
(GDP)}, the corresponding series name is \texttt{GDP}. The FRED web page
contains additional useful information such as the time period for which
the data is available, the data frequency (monthly, quarterly, annual)
and whether it's seasonally adjusted.

    \begin{tcolorbox}[breakable, size=fbox, boxrule=1pt, pad at break*=1mm,colback=cellbackground, colframe=cellborder]
\prompt{In}{incolor}{18}{\boxspacing}
\begin{Verbatim}[commandchars=\\\{\}]
\PY{c+c1}{\PYZsh{} The convention is to import this library as web}
\PY{k+kn}{import} \PY{n+nn}{pandas\PYZus{}datareader}\PY{n+nn}{.}\PY{n+nn}{data} \PY{k}{as} \PY{n+nn}{web}

\PY{c+c1}{\PYZsh{} define start and end dates}
\PY{n}{start\PYZus{}date} \PY{o}{=} \PY{l+s+s2}{\PYZdq{}}\PY{l+s+s2}{2000\PYZhy{}01\PYZhy{}01}\PY{l+s+s2}{\PYZdq{}}
\PY{n}{end\PYZus{}date} \PY{o}{=} \PY{l+s+s2}{\PYZdq{}}\PY{l+s+s2}{2021\PYZhy{}12\PYZhy{}31}\PY{l+s+s2}{\PYZdq{}}

\PY{c+c1}{\PYZsh{} Specify series name as first and \PYZsq{}fred\PYZsq{} data source as second argument}
\PY{n}{gdp} \PY{o}{=} \PY{n}{web}\PY{o}{.}\PY{n}{DataReader}\PY{p}{(}\PY{l+s+s1}{\PYZsq{}}\PY{l+s+s1}{GDP}\PY{l+s+s1}{\PYZsq{}}\PY{p}{,} 
    \PY{n}{data\PYZus{}source}\PY{o}{=}\PY{l+s+s1}{\PYZsq{}}\PY{l+s+s1}{fred}\PY{l+s+s1}{\PYZsq{}}\PY{p}{,} 
    \PY{n}{start}\PY{o}{=}\PY{n}{start\PYZus{}date}\PY{p}{,} 
    \PY{n}{end}\PY{o}{=}\PY{n}{end\PYZus{}date}
\PY{p}{)}

\PY{c+c1}{\PYZsh{} Show first 3 observations}
\PY{n}{gdp}\PY{o}{.}\PY{n}{head}\PY{p}{(}\PY{l+m+mi}{3}\PY{p}{)}
\end{Verbatim}
\end{tcolorbox}

            \begin{tcolorbox}[breakable, size=fbox, boxrule=.5pt, pad at break*=1mm, opacityfill=0]
\prompt{Out}{outcolor}{18}{\boxspacing}
\begin{Verbatim}[commandchars=\\\{\}]
                  GDP
DATE
2000-01-01  10002.179
2000-04-01  10247.720
2000-07-01  10318.165
\end{Verbatim}
\end{tcolorbox}
        
    We can also fetch multiple series at the same time, for example the CPI
(\texttt{CPIAUCSL}) and the unemployment rate (\texttt{UNRATE}).

    \begin{tcolorbox}[breakable, size=fbox, boxrule=1pt, pad at break*=1mm,colback=cellbackground, colframe=cellborder]
\prompt{In}{incolor}{19}{\boxspacing}
\begin{Verbatim}[commandchars=\\\{\}]
\PY{c+c1}{\PYZsh{} without an explicit end argument, the latest available data}
\PY{c+c1}{\PYZsh{} will be retrieved}
\PY{n}{data} \PY{o}{=} \PY{n}{web}\PY{o}{.}\PY{n}{DataReader}\PY{p}{(}\PY{p}{[}\PY{l+s+s1}{\PYZsq{}}\PY{l+s+s1}{CPIAUCSL}\PY{l+s+s1}{\PYZsq{}}\PY{p}{,} \PY{l+s+s1}{\PYZsq{}}\PY{l+s+s1}{UNRATE}\PY{l+s+s1}{\PYZsq{}}\PY{p}{]}\PY{p}{,} 
    \PY{n}{data\PYZus{}source}\PY{o}{=}\PY{l+s+s1}{\PYZsq{}}\PY{l+s+s1}{fred}\PY{l+s+s1}{\PYZsq{}}\PY{p}{,} 
    \PY{n}{start}\PY{o}{=}\PY{l+s+s1}{\PYZsq{}}\PY{l+s+s1}{2020\PYZhy{}01\PYZhy{}01}\PY{l+s+s1}{\PYZsq{}}
\PY{p}{)}

\PY{n}{data}\PY{o}{.}\PY{n}{head}\PY{p}{(}\PY{l+m+mi}{3}\PY{p}{)}
\end{Verbatim}
\end{tcolorbox}

            \begin{tcolorbox}[breakable, size=fbox, boxrule=.5pt, pad at break*=1mm, opacityfill=0]
\prompt{Out}{outcolor}{19}{\boxspacing}
\begin{Verbatim}[commandchars=\\\{\}]
            CPIAUCSL  UNRATE
DATE
2020-01-01   259.037     3.5
2020-02-01   259.248     3.5
2020-03-01   258.124     4.4
\end{Verbatim}
\end{tcolorbox}
        

\hypertarget{fred-federal-reserve-economic-data-optional}{%
\subsubsection{FRED: Federal Reserve Economic Data
(optional)}\label{fred-federal-reserve-economic-data-optional}}

\href{https://fred.stlouisfed.org/}{FRED}, provided by the Federal
Reserve of St.~Louis, is one of the most important macroeconomic online
databases (at least for US-centric data).
\href{https://pypi.org/project/fredapi/}{\texttt{fredapi}} is a Python
API for the FRED data which provides a wrapper for the FRED web service
(see also the project's \href{https://github.com/mortada/fredapi}{GitHub
page}).

Before accessing FRED, you need to install \texttt{fredapi} into your
Python environment as follows:

\begin{Shaded}
\begin{Highlighting}[]
\ExtensionTok{pip}\NormalTok{ install {-}{-}no{-}deps fredapi}
\end{Highlighting}
\end{Shaded}

\emph{Important:}

\begin{itemize}
\tightlist
\item
  The \texttt{-\/-no-deps} argument might be required for Anaconda users
  as otherwise the conda-provided versions of \texttt{numpy} and
  \texttt{pandas} could be overwritten.
\item
  Anaconda users should \emph{not} use the \texttt{fredapi} package
  provided in \texttt{conda-forge} as at the time of this writing it is
  outdated and will not work.
\end{itemize}

To use FRED, you additional need an API key which can be requested at
\url{https://fred.stlouisfed.org/docs/api/api_key.html}. Unlike with
some other APIs we discuss below, it is not possible to make a request
without a key. Once you have a key, you can specify it in several ways:

\begin{enumerate}
\def\labelenumi{\arabic{enumi}.}
\item
  On your local machine, set the environment variable
  \texttt{FRED\_API\_KEY} to store the key and it will be picked up
  automatically. This only works if you run a Python environment
  locally.
\item
  Store it in a file and pass the file name when creating a
  \texttt{Fred} instance:

\begin{Shaded}
\begin{Highlighting}[]
\ImportTok{from}\NormalTok{ fredapi }\ImportTok{import}\NormalTok{ Fred}
\NormalTok{fred }\OperatorTok{=}\NormalTok{ Fred(api\_key\_file}\OperatorTok{=}\StringTok{\textquotesingle{}path\_to\_file\textquotesingle{}}\NormalTok{)}
\end{Highlighting}
\end{Shaded}
\item
  Pass the string containing the API key as a parameter:

\begin{Shaded}
\begin{Highlighting}[]
\ImportTok{from}\NormalTok{ fredapi }\ImportTok{import}\NormalTok{ Fred}
\NormalTok{fred }\OperatorTok{=}\NormalTok{ Fred(api\_key}\OperatorTok{=}\StringTok{\textquotesingle{}INSERT API KEY HERE\textquotesingle{}}\NormalTok{)}
\end{Highlighting}
\end{Shaded}
\end{enumerate}

The following code assumes that the \texttt{FRED\_API\_KEY} variable has
been set up and might not work in your environment if that is not the
case.

\vspace{1em}\emph{Example: Retrieve the 10-year real interest rate}

    \begin{tcolorbox}[breakable, size=fbox, boxrule=1pt, pad at break*=1mm,colback=cellbackground, colframe=cellborder]
\prompt{In}{incolor}{20}{\boxspacing}
\begin{Verbatim}[commandchars=\\\{\}]
\PY{c+c1}{\PYZsh{} Uncomment the following to install fredapi in your local or }
\PY{c+c1}{\PYZsh{} cloud\PYZhy{}hosted Python environment (\eg Google Colab)}

\PY{c+c1}{\PYZsh{}! pip install \PYZhy{}\PYZhy{}no\PYZhy{}deps fredapi}
\end{Verbatim}
\end{tcolorbox}

    In order to retrieve any data, we first need to identify the series
name. This is easiest done by searching for the data on
\href{https://fred.stlouisfed.org/}{FRED} using your browser and copying
the series name, highlighted in red in the screenshot below.

\begin{figure}
\centering
\includegraphics{FRED_series_name.png}
\caption{FRED series name}
\end{figure}

For example, if we want to retrieve the
\href{https://fred.stlouisfed.org/series/REAINTRATREARAT10Y}{10-year
real interest rate}, the corresponding series name is
\texttt{REAINTRATREARAT10Y}. The FRED web page contains additional
useful information such as the time period for which the data is
available, the data frequency (monthly, quarterly, annual) and whether
it's seasonally adjusted.

    To download and plot the 10-year real interest rate, we proceed as
follows:

    \begin{tcolorbox}[breakable, size=fbox, boxrule=1pt, pad at break*=1mm,colback=cellbackground, colframe=cellborder]
\prompt{In}{incolor}{21}{\boxspacing}
\begin{Verbatim}[commandchars=\\\{\}]
\PY{k+kn}{from} \PY{n+nn}{fredapi} \PY{k+kn}{import} \PY{n}{Fred}

\PY{c+c1}{\PYZsh{} Create instance assuming API key is stored as environment variable}
\PY{n}{fred} \PY{o}{=} \PY{n}{Fred}\PY{p}{(}\PY{p}{)}

\PY{c+c1}{\PYZsh{} or specify API key directly}
\PY{c+c1}{\PYZsh{} fred = Fred(api\PYZus{}key=\PYZsq{}INSERT API KEY HERE\PYZsq{})}

\PY{c+c1}{\PYZsh{} Download observations starting from the year 2000 onward}
\PY{n}{series} \PY{o}{=} \PY{n}{fred}\PY{o}{.}\PY{n}{get\PYZus{}series}\PY{p}{(}\PY{l+s+s1}{\PYZsq{}}\PY{l+s+s1}{REAINTRATREARAT10Y}\PY{l+s+s1}{\PYZsq{}}\PY{p}{,} 
    \PY{n}{observation\PYZus{}start}\PY{o}{=}\PY{l+s+s1}{\PYZsq{}}\PY{l+s+s1}{2000\PYZhy{}01\PYZhy{}01}\PY{l+s+s1}{\PYZsq{}}
\PY{p}{)}
\end{Verbatim}
\end{tcolorbox}

    \begin{tcolorbox}[breakable, size=fbox, boxrule=1pt, pad at break*=1mm,colback=cellbackground, colframe=cellborder]
\prompt{In}{incolor}{22}{\boxspacing}
\begin{Verbatim}[commandchars=\\\{\}]
\PY{c+c1}{\PYZsh{} Print first 5 observations}
\PY{n}{series}\PY{o}{.}\PY{n}{head}\PY{p}{(}\PY{l+m+mi}{5}\PY{p}{)}
\end{Verbatim}
\end{tcolorbox}

            \begin{tcolorbox}[breakable, size=fbox, boxrule=.5pt, pad at break*=1mm, opacityfill=0]
\prompt{Out}{outcolor}{22}{\boxspacing}
\begin{Verbatim}[commandchars=\\\{\}]
2000-01-01    3.411051
2000-02-01    3.513343
2000-03-01    3.440347
2000-04-01    3.202967
2000-05-01    3.360531
dtype: float64
\end{Verbatim}
\end{tcolorbox}
        
    The data is returned as a pandas \texttt{Series} object with the
corresponding dates set as the index.

    \begin{tcolorbox}[breakable, size=fbox, boxrule=1pt, pad at break*=1mm,colback=cellbackground, colframe=cellborder]
\prompt{In}{incolor}{23}{\boxspacing}
\begin{Verbatim}[commandchars=\\\{\}]
\PY{n}{series}\PY{o}{.}\PY{n}{info}\PY{p}{(}\PY{p}{)}
\end{Verbatim}
\end{tcolorbox}

    \begin{Verbatim}[commandchars=\\\{\}]
<class 'pandas.core.series.Series'>
DatetimeIndex: 281 entries, 2000-01-01 to 2023-05-01
Series name: None
Non-Null Count  Dtype
--------------  -----
281 non-null    float64
dtypes: float64(1)
memory usage: 4.4 KB
    \end{Verbatim}

    \begin{tcolorbox}[breakable, size=fbox, boxrule=1pt, pad at break*=1mm,colback=cellbackground, colframe=cellborder]
\prompt{In}{incolor}{24}{\boxspacing}
\begin{Verbatim}[commandchars=\\\{\}]
\PY{c+c1}{\PYZsh{} Plot interest rate time series}
\PY{n}{series}\PY{o}{.}\PY{n}{plot}\PY{p}{(}\PY{n}{ylabel}\PY{o}{=}\PY{l+s+s1}{\PYZsq{}}\PY{l+s+s1}{10\PYZhy{}year real interest rate}\PY{l+s+s1}{\PYZsq{}}\PY{p}{)}
\end{Verbatim}
\end{tcolorbox}

            \begin{tcolorbox}[breakable, size=fbox, boxrule=.5pt, pad at break*=1mm, opacityfill=0]
\prompt{Out}{outcolor}{24}{\boxspacing}
\begin{Verbatim}[commandchars=\\\{\}]
<Axes: ylabel='10-year real interest rate'>
\end{Verbatim}
\end{tcolorbox}
        
    \begin{center}
    \adjustimage{max size={0.9\linewidth}{0.9\paperheight}}{unit07_files/unit07_53_1.pdf}
    \end{center}
    
    Other popular time series available on FRED are the
\href{https://fred.stlouisfed.org/series/CPIAUCSL}{CPI},
\href{https://fred.stlouisfed.org/series/GDPC1}{real GDP} and the
\href{https://fred.stlouisfed.org/series/UNRATE}{unemployment rate}.

    \hypertarget{nasdaq-data-api-optional}{%
\subsubsection{NASDAQ data API
(optional)}\label{nasdaq-data-api-optional}}

The NASDAQ stock exchange provides an open-source Python library hosted
on \href{https://github.com/Nasdaq/data-link-python}{GitHub} to access
various types of financial data (not only those traded on NASDAQ), see
\href{https://data.nasdaq.com/}{here} for details. The detailed API
documentation can be found at
\href{https://docs.data.nasdaq.com/docs}{here}. This data API was
formerly known as \href{https://github.com/quandl/quandl-python}{quandl}
which is no longer actively maintained but might still work.

Before using this service, you need to make sure that the Python package
is installed. Depending on how you launched this notebook, you may need
to execute the following code to install \texttt{nasdaq-data-link}:

\begin{Shaded}
\begin{Highlighting}[]
\ExtensionTok{pip}\NormalTok{ install nasdaq{-}data{-}link}
\end{Highlighting}
\end{Shaded}

Various types of data are available via this service and can be found
using the online search at \url{https://data.nasdaq.com/search}.

\begin{itemize}
\item
  Data come from various data provides. To select a data set, you
  usually have to specify a string of the form
  \texttt{\textquotesingle{}PROVIDER/SERIES\textquotesingle{}} where
  \texttt{\textquotesingle{}PROVIDER\textquotesingle{}} is the name of
  the provider (\eg \texttt{\textquotesingle{}FRED\textquotesingle{}}
  or \texttt{\textquotesingle{}BOE\textquotesingle{}}) and
  \texttt{\textquotesingle{}SERIES\textquotesingle{}} is the name of the
  time series.
\item
  Most of these data require a subscription or at least a free NASDAQ
  account. Once you have an account, you will need to get an API key and
  specify it when retrieving data. See the above links for details.
\item
  Some commercial data series include sample data that can be used
  without a subscription but requires a free NASDAQ account.
\item
  Some data series are freely available without a subscription or an
  account. These are often taken from other freely available data sets
  such as \href{https://fred.stlouisfed.org/}{FRED} or
  \href{https://www.blockchain.com/}{blockchain.com}. We'll be using
  these to demonstrate how the API works.

  \emph{Important:} Even for freely available data, NASDAQ imposes a cap
  of 50 web requests per day. You need to register to get around this.
\end{itemize}

The data is returned as pandas \texttt{DataFrame} object (or
alternatively as an NumPy array).

\vspace{1em}\emph{Example: Data from the Bank of England}

Let's start by retrieving some macroeconomic times series from the Bank
of England (BOE). It's not always straightforward to find the name of
the time series one is looking for, but you can see some of the
available time series
\href{https://data.nasdaq.com/data/BOE-bank-of-england-official-statistics/documentation}{here}.
The name will vary depending on the type of data (interest rate,
exchange rate), the frequency and how it is aggregated (daily, last day
of the month, monthly average) and a currency pair, if applicable.

    \begin{tcolorbox}[breakable, size=fbox, boxrule=1pt, pad at break*=1mm,colback=cellbackground, colframe=cellborder]
\prompt{In}{incolor}{25}{\boxspacing}
\begin{Verbatim}[commandchars=\\\{\}]
\PY{c+c1}{\PYZsh{} When running via Google Colab, uncomment and execute the following line}
\PY{c+c1}{\PYZsh{}! pip install nasdaq\PYZhy{}data\PYZhy{}link}
\end{Verbatim}
\end{tcolorbox}

    \begin{tcolorbox}[breakable, size=fbox, boxrule=1pt, pad at break*=1mm,colback=cellbackground, colframe=cellborder]
\prompt{In}{incolor}{26}{\boxspacing}
\begin{Verbatim}[commandchars=\\\{\}]
\PY{c+c1}{\PYZsh{} Retrieve 5\PYZhy{}year real implied yield on UK government bonds}
\PY{k+kn}{import} \PY{n+nn}{nasdaqdatalink} \PY{k}{as} \PY{n+nn}{ndl}
\PY{n}{df} \PY{o}{=} \PY{n}{ndl}\PY{o}{.}\PY{n}{get}\PY{p}{(}\PY{l+s+s1}{\PYZsq{}}\PY{l+s+s1}{BOE/IUMASRIF}\PY{l+s+s1}{\PYZsq{}}\PY{p}{)}

\PY{c+c1}{\PYZsh{} Rename column which is always called \PYZsq{}Value\PYZsq{}}
\PY{n}{df} \PY{o}{=} \PY{n}{df}\PY{o}{.}\PY{n}{rename}\PY{p}{(}\PY{n}{columns}\PY{o}{=}\PY{p}{\PYZob{}}\PY{l+s+s1}{\PYZsq{}}\PY{l+s+s1}{Value}\PY{l+s+s1}{\PYZsq{}}\PY{p}{:} \PY{l+s+s1}{\PYZsq{}}\PY{l+s+s1}{Yield}\PY{l+s+s1}{\PYZsq{}}\PY{p}{\PYZcb{}}\PY{p}{)}

\PY{c+c1}{\PYZsh{} Plot time series }
\PY{n}{df}\PY{o}{.}\PY{n}{plot}\PY{p}{(}\PY{n}{ylabel}\PY{o}{=}\PY{l+s+s1}{\PYZsq{}}\PY{l+s+s1}{5 Year Real Implied Forward (Monthly)}\PY{l+s+s1}{\PYZsq{}}\PY{p}{)}
\end{Verbatim}
\end{tcolorbox}

            \begin{tcolorbox}[breakable, size=fbox, boxrule=.5pt, pad at break*=1mm, opacityfill=0]
\prompt{Out}{outcolor}{26}{\boxspacing}
\begin{Verbatim}[commandchars=\\\{\}]
<Axes: xlabel='Date', ylabel='5 Year Real Implied Forward (Monthly)'>
\end{Verbatim}
\end{tcolorbox}
        
    \begin{center}
    \adjustimage{max size={0.9\linewidth}{0.9\paperheight}}{unit07_files/unit07_58_1.pdf}
    \end{center}
    
    As another example, we retrieve the US dollar / Sterling exchange rate
at a monthly frequency (this is determined by the name of the time
series used where \texttt{ML} requests the monthly series, using the
last observation for each month). Note that we can pass additional
arguments, for example restricting the time period we want to retrieve
using \texttt{start\_date} and \texttt{end\_date}.

    \begin{tcolorbox}[breakable, size=fbox, boxrule=1pt, pad at break*=1mm,colback=cellbackground, colframe=cellborder]
\prompt{In}{incolor}{27}{\boxspacing}
\begin{Verbatim}[commandchars=\\\{\}]
\PY{c+c1}{\PYZsh{} Get USD / GDP exchange rate using the last observation for each month.}
\PY{n}{df} \PY{o}{=} \PY{n}{ndl}\PY{o}{.}\PY{n}{get}\PY{p}{(}\PY{l+s+s1}{\PYZsq{}}\PY{l+s+s1}{BOE/XUMLUSS}\PY{l+s+s1}{\PYZsq{}}\PY{p}{,} \PY{n}{start\PYZus{}date}\PY{o}{=}\PY{l+s+s1}{\PYZsq{}}\PY{l+s+s1}{2000\PYZhy{}01\PYZhy{}31}\PY{l+s+s1}{\PYZsq{}}\PY{p}{)}
\PY{n}{df} \PY{o}{=} \PY{n}{df}\PY{o}{.}\PY{n}{rename}\PY{p}{(}\PY{n}{columns}\PY{o}{=}\PY{p}{\PYZob{}}\PY{l+s+s1}{\PYZsq{}}\PY{l+s+s1}{Value}\PY{l+s+s1}{\PYZsq{}}\PY{p}{:} \PY{l+s+s1}{\PYZsq{}}\PY{l+s+s1}{USD\PYZus{}GBP}\PY{l+s+s1}{\PYZsq{}}\PY{p}{\PYZcb{}}\PY{p}{)}

\PY{c+c1}{\PYZsh{} Plot USD/GBP time series}
\PY{n}{df}\PY{o}{.}\PY{n}{plot}\PY{p}{(}\PY{n}{ylabel}\PY{o}{=}\PY{l+s+s1}{\PYZsq{}}\PY{l+s+s1}{USD/GBP exchange rate}\PY{l+s+s1}{\PYZsq{}}\PY{p}{)}
\end{Verbatim}
\end{tcolorbox}

            \begin{tcolorbox}[breakable, size=fbox, boxrule=.5pt, pad at break*=1mm, opacityfill=0]
\prompt{Out}{outcolor}{27}{\boxspacing}
\begin{Verbatim}[commandchars=\\\{\}]
<Axes: xlabel='Date', ylabel='USD/GBP exchange rate'>
\end{Verbatim}
\end{tcolorbox}
        
    \begin{center}
    \adjustimage{max size={0.9\linewidth}{0.9\paperheight}}{unit07_files/unit07_60_1.pdf}
    \end{center}
    
\vspace{1em}\emph{Example: Data from blockchain.com}

The NASDAQ data link also supports retrieving data on cryptocurrencies.
For example, there is a freely accessible time series for the price of
Bitcoin in USD.

    \begin{tcolorbox}[breakable, size=fbox, boxrule=1pt, pad at break*=1mm,colback=cellbackground, colframe=cellborder]
\prompt{In}{incolor}{28}{\boxspacing}
\begin{Verbatim}[commandchars=\\\{\}]
\PY{k+kn}{import} \PY{n+nn}{nasdaqdatalink} \PY{k}{as} \PY{n+nn}{ndl}

\PY{c+c1}{\PYZsh{} Retrieve price of BTC in USD for 2022}
\PY{n}{df} \PY{o}{=} \PY{n}{ndl}\PY{o}{.}\PY{n}{get}\PY{p}{(}\PY{l+s+s1}{\PYZsq{}}\PY{l+s+s1}{BCHAIN/MKPRU}\PY{l+s+s1}{\PYZsq{}}\PY{p}{,} \PY{n}{start\PYZus{}date}\PY{o}{=}\PY{l+s+s1}{\PYZsq{}}\PY{l+s+s1}{2022\PYZhy{}01\PYZhy{}31}\PY{l+s+s1}{\PYZsq{}}\PY{p}{)}

\PY{c+c1}{\PYZsh{} Change column name to something more descriptive}
\PY{n}{df} \PY{o}{=} \PY{n}{df}\PY{o}{.}\PY{n}{rename}\PY{p}{(}\PY{n}{columns}\PY{o}{=}\PY{p}{\PYZob{}}\PY{l+s+s1}{\PYZsq{}}\PY{l+s+s1}{Value}\PY{l+s+s1}{\PYZsq{}}\PY{p}{:} \PY{l+s+s1}{\PYZsq{}}\PY{l+s+s1}{BTC\PYZus{}USD}\PY{l+s+s1}{\PYZsq{}}\PY{p}{\PYZcb{}}\PY{p}{)}

\PY{c+c1}{\PYZsh{} Plot time series}
\PY{n}{df}\PY{o}{.}\PY{n}{plot}\PY{p}{(}\PY{n}{ylabel}\PY{o}{=}\PY{l+s+s1}{\PYZsq{}}\PY{l+s+s1}{BTC price in USD}\PY{l+s+s1}{\PYZsq{}}\PY{p}{)}
\end{Verbatim}
\end{tcolorbox}

            \begin{tcolorbox}[breakable, size=fbox, boxrule=.5pt, pad at break*=1mm, opacityfill=0]
\prompt{Out}{outcolor}{28}{\boxspacing}
\begin{Verbatim}[commandchars=\\\{\}]
<Axes: xlabel='Date', ylabel='BTC price in USD'>
\end{Verbatim}
\end{tcolorbox}
        
    \begin{center}
    \adjustimage{max size={0.9\linewidth}{0.9\paperheight}}{unit07_files/unit07_62_1.pdf}
    \end{center}
    
\vspace{1em}\emph{Example: Historical stock data}

As a final example, we obtain the trading data for the stock of Apple
(ticker symbol \texttt{AAPL}) for the year 2001. Such data is often not
available without a subscription or a login, but it works if the
requested time period is sufficiently far in the past!

    \begin{tcolorbox}[breakable, size=fbox, boxrule=1pt, pad at break*=1mm,colback=cellbackground, colframe=cellborder]
\prompt{In}{incolor}{29}{\boxspacing}
\begin{Verbatim}[commandchars=\\\{\}]
\PY{c+c1}{\PYZsh{} Retrieve stock data for Apple (ticker symbol AAPL)}
\PY{n}{df} \PY{o}{=} \PY{n}{ndl}\PY{o}{.}\PY{n}{get}\PY{p}{(}\PY{l+s+s2}{\PYZdq{}}\PY{l+s+s2}{WIKI/AAPL}\PY{l+s+s2}{\PYZdq{}}\PY{p}{,} 
    \PY{n}{start\PYZus{}date}\PY{o}{=}\PY{l+s+s1}{\PYZsq{}}\PY{l+s+s1}{2000\PYZhy{}01\PYZhy{}01}\PY{l+s+s1}{\PYZsq{}}\PY{p}{,} 
    \PY{n}{end\PYZus{}date}\PY{o}{=}\PY{l+s+s1}{\PYZsq{}}\PY{l+s+s1}{2000\PYZhy{}12\PYZhy{}31}\PY{l+s+s1}{\PYZsq{}}
\PY{p}{)}
\end{Verbatim}
\end{tcolorbox}

    Unlike in the previous examples, this data contains not only a single
value, but a whole range of variables including the opening and closing
price, the trading volume, \etc:

    \begin{tcolorbox}[breakable, size=fbox, boxrule=1pt, pad at break*=1mm,colback=cellbackground, colframe=cellborder]
\prompt{In}{incolor}{30}{\boxspacing}
\begin{Verbatim}[commandchars=\\\{\}]
\PY{n}{df}\PY{o}{.}\PY{n}{info}\PY{p}{(}\PY{p}{)}
\end{Verbatim}
\end{tcolorbox}

    \begin{Verbatim}[commandchars=\\\{\}]
<class 'pandas.core.frame.DataFrame'>
DatetimeIndex: 252 entries, 2000-01-03 to 2000-12-29
Data columns (total 12 columns):
 \#   Column       Non-Null Count  Dtype
---  ------       --------------  -----
 0   Open         252 non-null    float64
 1   High         252 non-null    float64
 2   Low          252 non-null    float64
 3   Close        252 non-null    float64
 4   Volume       252 non-null    float64
 5   Ex-Dividend  252 non-null    float64
 6   Split Ratio  252 non-null    float64
 7   Adj. Open    252 non-null    float64
 8   Adj. High    252 non-null    float64
 9   Adj. Low     252 non-null    float64
 10  Adj. Close   252 non-null    float64
 11  Adj. Volume  252 non-null    float64
dtypes: float64(12)
memory usage: 25.6 KB
    \end{Verbatim}

    To plot a specific column, we can use the \texttt{y=...} argument to
\texttt{DataFrame.plot()}.

    \begin{tcolorbox}[breakable, size=fbox, boxrule=1pt, pad at break*=1mm,colback=cellbackground, colframe=cellborder]
\prompt{In}{incolor}{31}{\boxspacing}
\begin{Verbatim}[commandchars=\\\{\}]
\PY{n}{df}\PY{o}{.}\PY{n}{plot}\PY{p}{(}\PY{n}{y}\PY{o}{=}\PY{l+s+s1}{\PYZsq{}}\PY{l+s+s1}{Close}\PY{l+s+s1}{\PYZsq{}}\PY{p}{,} \PY{n}{ylabel}\PY{o}{=}\PY{l+s+s1}{\PYZsq{}}\PY{l+s+s1}{Stock prive of AAPL}\PY{l+s+s1}{\PYZsq{}}\PY{p}{)}
\end{Verbatim}
\end{tcolorbox}

            \begin{tcolorbox}[breakable, size=fbox, boxrule=.5pt, pad at break*=1mm, opacityfill=0]
\prompt{Out}{outcolor}{31}{\boxspacing}
\begin{Verbatim}[commandchars=\\\{\}]
<Axes: xlabel='Date', ylabel='Stock prive of AAPL'>
\end{Verbatim}
\end{tcolorbox}
        
    \begin{center}
    \adjustimage{max size={0.9\linewidth}{0.9\paperheight}}{unit07_files/unit07_68_1.pdf}
    \end{center}
    

    % Add a bibliography block to the postdoc
    
    
    
\end{document}
